This is a reference manual for command usage in Latex for the IEEEtran class.
It is based on the reference manual: % https://mirrors.up.pt/pub/CTAN/macros/latex/contrib/IEEEtran/IEEEtran_HOWTO.pdf

%%%%%%%%%%%%%%%%%%%%%%%%%%% Workflow %%%%%%%%%%%%%%%%%%%%%%%%%%%%%%%%%%%%%
dissertation.tex is the main file. 
There are inputted the secondary files for writing, representing the main logical
division of the document.
Additional logical division can be performed by inputting files in the secondary files,
i.e. nesting, which may be useful in larger projects.

Thus, a suitable workflow can be:
1) Write out your part in any of the secondary files, adding a new sectioning command (see Sections)
2) If your part is too big, you may prefer to input it as new secondary file inside a main secondary file.
3) Define symbols and acronyms as suitable, so they can be easily referenced inside the doc (see Acronyms and Symbols)
4) Always label all relevant elements (see References)
5) Cite other people's work (see Citations and Bibliography)
%%%%%%%%%%%%%%%%%%%%%%%%%%%%%%%%%%%%%%%%%%%%%%%%%%%%%%%%%%%%%%%%%%%%%%%%%%

%%%%%%%%%%%%%%%%%%%%%%%%%%% Sections %%%%%%%%%%%%%%%%%%%%%%%%%%%%%%%%%%%%
Sections are the highest sectioning command available for the IEEEtran class,
enabling the division of the document into logical portions.

The IEEEtran allows:
- section: % \section{Analysis}%
- subsection
- subsubsection
- paragraph
- subparagraph

If a starred environment is used, e.g., section*, it will not be numbered,
unlike the default section.

Example:
% SECTION - Introduction -------------------------
\section{Introduction}%
\label{sec:introduction}%
%%%%%%%%%%%%%%%%%%%%%%%%%%%%%%%%%%%%%%%%%%%%%%%%%%%%%%%%%%%%%%%%%%%%%%%%%

%%%%%%%%%%%%%%%%%%%%%%%%%% References %%%%%%%%%%%%%%%%%%%%%%%%%%%%%%%%%%%
References are a mechanism to visit key places in the document, such as:
- figures
- tables
- sections
- listings

References require a label (where to jump) and a reference (invoking the jump).

They should be added to all these elements, i.e. labelled, using:
% \label{<elem>:name}
% where <elem> can be: sec, fig, tab, lst

To reference it in another part of the text, use:
%\ref{<elem>:name}

NOTE: every element must be referenced in the text before it appears.

Example:
The \gls{imu} was attached to textile band, positioned in the lower body of an
individual, according to the axes orientation defined in Fig.~\ref{fig:fig2}.
% Fig 1
\begin{figure*} 
\centering
    \includegraphics[width=1.0\textwidth]{./img/fig2.png}
  \caption{System used for inertial
    data acquisition: orientation of the \gls{imu} during the data acquisition
    and orientation after the compensation; acquisition protocol (withdrawn from~\cite{cav-enunc})}% 
\label{fig:fig2}
\end{figure*}
%%%%%%%%%%%%%%%%%%%%%%%%%%%%%%%%%%%%%%%%%%%%%%%%%%%%%%%%%%%%%%%%%%%%%%%%

%%%%%%%%%%%%%%%%%%%%%%%%%% Acronyms and Symbols %%%%%%%%%%%%%%%%%%%%%%%%
Acronyms and Symbols are used to ease the writing:
- acronyms refer to abbreviations of concepts
- symbols refer to mathematical/physical symbols

Together, they are used to build a glossary.

Acronyms are:
- defined in /sec/acronyms.tex
- referenced in text, using: \gls{<acr_key>}, e.g., \gls{imu}

Symbols are:
- defined in /sec/symbols.tex
- referenced in text, using: \gls{<symb_key>}, e.g., \gls{varphi-a}
%%%%%%%%%%%%%%%%%%%%%%%%%%%%%%%%%%%%%%%%%%%%%%%%%%%%%%%%%%%%%%%%%%%%%%%%

%%%%%%%%%%%%%%%%%%%%%%%%% Citations and Bibliography %%%%%%%%%%%%%%%%%%%%
Citations are used to give credits to someone else's work. 
It requires a bibliography file, containing the citations' keys, and a 
citation to that key.

Usage:
1. Define the bibliography entry in /bib/dissert.bib. The easiest way to do this is:
  1) Search the work in Google scholar: https://scholar.google.pt/
  2) Click on the " (Citar) below the search entry
  3) Select Bibtex
  4) Copy the entry and place it in the bib file.
2. Cite using: %\cite{bibKey}

Example:
% Bib file
%@article{mccarron2013low,
%  title={Low-cost IMU implementation via sensor fusion algorithms in the Arduino environment},
%  author={McCarron, Brandon},
%  year={2013}
%}
... as they only sense these variations and do not have a fixed frame of
reference~\cite{mccarron2013low}.
%%%%%%%%%%%%%%%%%%%%%%%%%%%%%%%%%%%%%%%%%%%%%%%%%%%%%%%%%%%%%%%%%%%%%%%%%

%%%%%%%%%%%% Figures %%%%%%%%%%%%%%%%%%%%%%%%%
Images should be added to folder /img/.

% figure*: double-column
% figure: single column
% position [!hbt]: not used - class will select where to place the float

% double-column
\begin{figure*} 
\centering
    \includegraphics[width=1.0\textwidth]{./img/fig2.png}
  \caption{System used for inertial
    data acquisition: orientation of the \gls{imu} during the data acquisition
    and orientation after the compensation; acquisition protocol (withdrawn from~\cite{cav-enunc})}%
\label{fig:fig2}
\end{figure*}

% single-column
\begin{figure}
\centering
    \includegraphics[width=0.9\columnwidth]{./img/kalmanf.png}
  \caption{KF procedure overview (withdrawn from~\cite{mccarron2013low})}%
\label{fig:kalmanf}
\end{figure}

% subfigures
\begin{figure}[htb!]
  \centering
  %
  \begin{subfigure}{.4\textwidth}
  \includegraphics[width=\textwidth]{img/PTestOriginalPosition.jpg}%
  \caption{KUKA's original position}%
  \label{fig:ptp-test-orig}
\end{subfigure}
%
  \begin{subfigure}{.20\textwidth}
    \includegraphics[width=\textwidth]{img/PTestFinalPosition.jpg}%
  \caption{KUKA's final position}%
  \label{fig:ptp-test-final}
  \end{subfigure}
%
  \begin{subfigure}{.38\textwidth}
    \includegraphics[width=\textwidth]{img/PTestOutput.png}%
  \caption{PTP command output --- custom solution}%
  \label{fig:ptp-test-out}
  \end{subfigure}
  % 
  \caption{PTP command test --- custom solution}%
  \label{fig:ptp-test}
\end{figure}
%%%%%%%%%%%%%%%%%%%%%%%%%%%%%%%%%%%%%%%%%%%%%%%%%%%

%%%%%%%%%%%%%%% Tables %%%%%%%%%%%%%%%%%%%%%%%%%%%%%%%
% table is also a float, so it can be used similarly
% for table creation use: https://www.tablesgenerator.com/#
% table*: double-column
% table: single column
% position [!hbt]: not used - class will select where to place the float

% ADJUST SPACINGS
% src: https://www.overleaf.com/project/5c349976c042023b1bd97751
% A table with adjusted row and column spacings
% \setlength sets the horizontal (column) spacing
% \arraystretch sets the vertical (row) spacing
%\begingroup
%\setlength{\tabcolsep}{10pt} % Default value: 6pt
%\renewcommand{\arraystretch}{0.5} % Default value: 1
%\begin{tabular}{ c c c }
%First Row & -6 & -5 \\
%Second Row & 4 & 10\\
%Third Row & 20 & 30\\
%Fourth Row & 100 & -30\\
%\end{tabular}
%\endgroup
% The \begingroup ... \endgroup pair ensures the separation
% parameters only affect this particular table, and not any
% sebsequent ones in the document.

% double-column
\begin{table*}
   \centering
   \caption{Defined assumptions for KF}%
   \label{tab:parameter} 
  \begin{tabular}{|l|l|l|l|}
    \hline
  %\rowcolor[HTML]{DAE8FC} 
  Symbol & Definition                           & Value & Assumption                                                                                                   \\ \hline
  u\_k   & Control input                        & 0     & \begin{tabular}[c]{@{}l@{}}State of system does not\\  change between steps\end{tabular}                     \\ \hline
  H      & Relates current state to measurement & 1     & \begin{tabular}[c]{@{}l@{}}Measurement state is \\ equivalent current state\end{tabular}                     \\ \hline
  Q      & Process noise covariance             & 0.1   & \begin{tabular}[c]{@{}l@{}}Constant chosen to tune \\ filter as in ref. doc.7\end{tabular} \\ \hline
  R      & Measurement Noise                    & 5     & \begin{tabular}[c]{@{}l@{}}Driven by accelerometer \\ raw output jitter
                                                          \end{tabular}
  \\ \hline
\end{tabular}
\end{table*}
%%%%%%%%%%%%%%%%%%%%%%%%%%%%%%%%%%%%%%%%%%%%%%%%%%%%%%%%%%%%%%%%%%%%%%%%%%%%%%%%%

%%%%%%%%%%%%%%%%%%%%%%%%%%% Listings %%%%%%%%%%%%%%%%%%%%%%%%%%%%%%%%%%%%%%%%%%
Listings' style is defined in /sty/listing.sty.
- To define a new style, just copy the default one and change the appropriate parameters.

To include a listing:
1. Add the code to /listing/ (e.g., /listing/KF_main_short.m)
2. Include the listing in the text, using:
%
\lstinputlisting[language=matlab, firstline=17,caption={KF main implementation in \matlab},label=lst:kf-main-short,
style=custom-matlab]{./listing/KF_main_short.m}%
%%%%%%%%%%%%%%%%%%%%%%%%%%%%%%%%%%%%%%%%%%%%%%%%%%%%%%%%%%%%%%%%%%%%%%%%%%%%%%%%%%


