%
\section{Software specification}
\label{sec:sw-specs}
% Next, the \gls{sw} responsible for system operation is specified. As this system
% is an event-driven asynchronous one it can be more easily specified using a
% state machine diagram as illustrated in Fig.~\ref{fig:state-mach}. At program
% startup --- \texttt{Idle} state --- the peripherals are initialized (interrupts, PCA, I2C, timer for tick
% generated interrupts) and the program waits for events, going into sleep
% mode.
% 
% When a key is pressed, the external interrupt \texttt{INTO} signals this
% event right away to the CPU awaking it up triggering the execution of
% \texttt{ISR\_Key\_Press}. This ISR checks which key was pressed by reading it
% from the I/O expander via I2C, pushes the associated code to
% \texttt{keycode\_fifo} --- a circular \gls{fifo} buffer --- and flags it to the CPU, enabling \texttt{keycode\_avail}.
% 
% The CPU is awaked regularly to check if any keycode is available. If it is, it
% triggers the execution of \texttt{ISR\_IR\_emit}, clearing the flag
% \texttt{keycode\_avail}, popping the keycode from \texttt{keycode\_fifo} and
% sending it via IR circuitry (\texttt{ir\_send(keycode)}).
% %
%   \vspace{-5mm}
% %  
% \begin{figure}[htb!]
% \centering
%     \includegraphics[width=0.7\columnwidth]{./img/state-mach.png}
%   \caption{TV remote state machine diagram}%
% \label{fig:state-mach}
% \end{figure}
% 
% Fig.~\ref{fig:i2cread-flow} presents the flowchart for reading from the I/O expander via i2c
% 
% \begin{figure}[htb!]
% \centering
%     \includegraphics[width=0.7\columnwidth]{./img/i2c_read_flowchart.png}
%   \caption{I2C Read Flowchart}%
% \label{fig:i2cread-flow}
% \end{figure}
% %
%   \vspace{-5mm}
% %  
\section{Software interfaces definition}
\label{sec:sw-interf-def}
%- Define the APIs in detail:
%  - header files with:
%    - functions prototypes
%    - data structure declarations
%    - class declarations
%
% From the state machine diagram (Fig.~\ref{fig:state-mach}) the softwares modules
% and data structures can be infered. The data structure is \texttt{keycode\_fifo},
% a circular buffer to manage the keycodes produced and consumed. The modules are
% \texttt{i2c}, \texttt{ir} and \texttt{fifo} for determing the key pressed,
% transmitting the keycode and managing the \gls{fifo} buffer. Enumerations are
% added for keycodes and errors listing.
% 
% An example of the \gls{api} can be seen in Listing~\ref{lst:api}, using builtin documentation.
% %
% \lstinputlisting[language=c, firstline=1,caption={\gls{api} example with
%   builtin documentation},label=lst:api,
% style=customc]{./listing/api.h}%

\section{Start-up/shutdown process specification}
\label{sec:startup-shutdown}
%As highlighted in Fig.~\ref{fig:state-mach}, the process starts with battery
%power being supplied to the system, going into sleep mode and waiting for
%events, minimizing power consumption. However, there is still residual power
%being drawn. This could be overcome by placing a power button for the remote
%itself, but with the inconvenience of MCU reset and the initial delays
%associated. The shutdown results from batteries disconnection.\

\section{Error handling specification}
\label{sec:error-handling-specification}
%Every system is prone to glitches, bugs and errors, thus, requiring it to be
%handled. Errors should be handled gracefully by creating error handling
%routines, which try to circumvent them and provide feedback.
%
%For extreme cases, where this is not possible, the watchdog timer should be
%enabled to help the system recover from crashes. However, this is a last resort,
%as constant reboots --- sign of a bad design --- are inadmissible and will
%frustrate the user.
%
%The error handling routines can be built into the design by considering return
%codes and asserts from function calls, e.g., \texttt{int i2c\_read(char *byte)},
%where \texttt{0} signals success and otherwise an error was encountered. Thus,
%good design eases error-handling specification and should be an aspect to keep
%in mind in this phase.
%%
%  \vspace{-5mm}
%%% Local Variables:
%%% mode: latex
%%% TeX-master: "../../../dissertation"
%%% End:
