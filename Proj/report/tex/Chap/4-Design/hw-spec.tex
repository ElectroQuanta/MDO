%
\section{Hardware specification}
\label{sec:hw-specs}
% The first step for system design is the \gls{hw} specification. This can be
% pictured as a block diagram, ideally with \gls{cots}
% components. Fig.~\ref{fig:block-diag} depicts the overall information flow and
% the TV remote block diagram: the \texttt{User} interacts with
% the TV remote by pressing buttons which triggers the emission of \gls{ir} codes
% to the TV, containing the \gls{ir} receiver. As supported by the market research
% seen in
% Section~\ref{sec:market-research}, the block diagram of the TV remote control
% can be taught of a physical input interface --- the pushbuttons, a processing
% interface --- a microprocessor --- to determine which keys are pressed and the
% resulting \gls{ir} codes to be emitted, and a \gls{ir} transmitter circuit as
% the output where the IR LED also works as a visual output. Usually, there will
% be also a digital interface, to handle the large amount of keys as a
% \gls{io} expander with serial interface, e.g. the Microchip MCP23008/MCP23S08~\cite{microchip}, which is considered here for future expansibility of the device,
% thus making it scalable. This can be especially useful when using
% microprocessor units with lower I/O inputs. A tradeoff between the inclusion of
% the multiplexer versus future redesign must be performed. The system is powered
% by batteries.
% 
% The microprocessor chosen depends on various factors, such as: architecture,
% throughput, memory, processing power, power efficiency, toolchain availability
% and programming easiness, etc. Additionally, one may consider the usage of a
% microcontroller, due to added peripherals, easing the \gls{sw} burden. For
% example, the usage of the \gls{pca} peripheral in the 8051 \gls{mcu} can free the processor from the
% job of bit-banging the IR transmitter circuitry or the \gls{i2c} or \gls{spi} peripherals to
% interface the I/O expander. With that said, the choice
% relied on the 8051 MCU due to its small footprint, clock speed of up to 48 MHz,
% low power usage on idle, the acquaintance with the toolchain and the
% programming, and for the inclusion of the PCA, SPI and I2C peripherals.  
% %
%   \vspace{-5mm}
% %  
% %- Block diagram with COTS components, if possible
% %- List of constraints of functions to be implemented in HW or SW
% %  - Inclusion of a multiplexer may reduce SW burden
% %  - CPU peripherals:
% %  - PCA for wave generation
% %  
% \begin{figure}[htb!]
% \centering
%     \includegraphics[width=0.7\columnwidth]{./img/block-diagram.png}
%   \caption{Overall information flow and TV remote block diagram}%
% \label{fig:block-diag}
% \end{figure}
% %
%   \vspace{-5mm}
% %  
\section{Hardware interfaces definition}
\label{sec:hw-interf-def}
% After specifying the \gls{hw}, it is important to define its interfaces. The
% TV remote control is clearly an event-driven asynchronous system, thus using HW
% interrupts. When a user presses a pushbutton that event should be signaled to
% the \gls{cpu} of the MCU via an external interrupt, ``waking'' it up. The CPU
% handles that interrupt via an \gls{isr}, processing it and actuating, if
% required. The system then goes back to sleep.
% 
% The 8051 MCU only contains 2 external interrupts sources --- \texttt{INTO} and
% \texttt{INT1} --- thus limitting the direct connection of the pushbuttons to the
% MCU, even with the pull-up circuitry. This is where the serial I/O expander
% becomes most useful, as it can be connected to TV remote keys (up to 8 in this
% case) and connected to a single external interrupt pin, being then read via
% \gls{i2c} or \gls{spi} interface. Thus, the keys can be connected through
% pull-up circuitry to the I/O expander and the output is then connected to
% \texttt{INTO} on the 8051 MCU. The communication protocol chosen is \gls{i2c} as
% it is more expandable
% 
% Concerning the output, the 8051 PCA peripheral is connected to the IR
% transmitter circuitry, for IR code emission. The HW interfaces of the system are
% depicted in Fig.~\ref{fig:hw-interfaces}.
% %
%   \vspace{-5mm}
% %  
% \begin{figure}[htb!]
% \centering
%     \includegraphics[width=1.0\columnwidth]{./img/hw-interfaces.jpg}
%   \caption{HW interfaces of the system}%
% \label{fig:hw-interfaces}
% \end{figure}
% %
% %
%   \vspace{-5mm}
%  
%%% Local Variables:
%%% mode: latex
%%% TeX-master: "../../../dissertation"
%%% End:
