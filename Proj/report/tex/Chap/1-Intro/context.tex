\section{Context and motivation}
\label{sec:context-motivation}
COVID pandemics presented a landmark on human interaction, greatly reducing the
contact between people and surfaces. Thus, it is an imperative to provide people
with contactless interfaces for everyday tasks. People redefined their
purchasing behaviors, leading to a massive growth of the online
shopping. However, some business sectors, like clothing or perfumes, cannot
provide the same user experience when moving online.
Therefore, one proposes to close that gap by providing a marketing digital
outdoor for brands to advertise and gather customers with contactless
interaction.

Scenting marketing is a great approach to draw people into stores.
Olfactory sense is the fastest way to the brain, thus, providing an exceptional
opportunity for marketing~\cite{news-harvard} --- ``75\% of the emotions we generate on a daily basis are affected by smell. Next
to sight, it is the most important sense we have''~\cite{lindstrom2006brand}.

Combining that with additional stimuli, like sight and sound, can
significantly boost the marketing outcome. Brands can buy advertisement space
and time, selecting the videoclips to be displayed and the fragrance to be
used at specific times, drawing the customers into their stores.

Marketing also leverages from better user experience, thus, user interaction is
a must-have, providing the opportunity to interact with the customer. In this
sense, when users approach the outdoor a gesture-based interface will be
provided for a brand immersive experience, where the user can take pictures or
create GIFs with brand specific image filters and share them through their
social media, with the opportunity to gain several benefits.

%%% Local Variables:
%%% mode: latex
%%% TeX-master: "../../../dissertation"
%%% End:
