\section{Project planning}%
\label{sec:project-planning}
%
In Appendix~\ref{ch:append-gantt-diag} is illustrated the Gantt chart for the
project (Fig.~\ref{fig:gantt-diag-orig}), containing the tasks' descriptions. It
should be noted that the project follows the Waterfall project
methodology, which is meant to be iterative.

The tasks are described as follows:
\begin{item-c}
\item \emph{Project planning}: in the project planning, a brainstorming about conceivable devices takes place, whose
viability is then assessed, resulting in the problem statement
(Milestone 0). A market research is performed to assess the product's market
space and opportunities. Finally, an initial version of the project planning is
conceived to define a feasible timeline for the suggested tasks.
%
\item \emph{Analysis}: in this phase an overview of the system is conceived,
  presenting a global picture of the problem and a viable solution. The
  requirements and constraints are the elicited, defining the required features
  and environmental restrictions on the solution. The system architecture is
  then derived and subsequently decomposed into subsystems to ease the
  development, consisting of the events, use cases, dynamic operation of the
  system and the flow of events throughout the system. Finally, the theoretical
  foundations for the project development are presented.
%
\item \emph{Design}: at this stage the analysis specification is reviewed, and
  the \gls{hw} and \gls{sw} and the respective interfaces are fully
  specified. The \gls{hw} specification yields the respective document, enabling
  the component selection, preferably \gls{cots}, and shipping. The \gls{sw} specification is separately performed in the
  subsystems identified, yielding the \gls{sw} specifications documentation (milestone).
%
\item \emph{Implementation}: product implementation which is done by
  \emph{modular integration}. The HW is tested and the SW is implemented in the
  target platforms, yielding the SW source code as a deliverable (milestone).
  The designed HW circuits are then tested in breadboards for verification and
  the corresponding \gls{pcb} is designed, manufactured and assembled. Lastly,
  the system configuration is performed, yielding prototype alpha of the product.
%
\item \emph{Tests}: modular tests and integrated tests are performed regarding
  the HW and SW components and a functional testing is conducted.
\item \emph{Functional Verification/Validation}: System verification is
  conducted to validate overall function. Regarding validation, it is conducted
  by an external agent, where a user should try to interact with the designed prototype.
\item \emph{Documentation}: throughout the project the several phases will be
  documented, comprising several milestones, namely: problem statement; analysis; design; implementation; and
  final.
\end{item-c}
%%% Local Variables:
%%% mode: latex
%%% TeX-master: "../../../dissertation"
%%% End:
