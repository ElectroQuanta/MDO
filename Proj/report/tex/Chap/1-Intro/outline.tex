\section{Report Outline}%
\label{sec:report-outline}
This report is organized as follows:
\begin{itemize}
\item In Chapter~\ref{ch:introduction} is presented the project's context and
  motivation, the problem statement, the market research, the project goals, and
  project planning.
\item In Chapter~\ref{ch:analysis}, the product requirements are derived --- defining the client expectations
for the product --- as well as the project constraints --- what the environments
limits about the product. Based on the set of requirements and constraints, a
system overview is produced, capturing the main features and interactions with
the system, as well as its key components.
Then, the system architecture is
devised, comprising both hardware and software domains. Next, the system is
decomposed into subsystems, presenting a deeper analysis over it, comprising its
user mockups, events, use cases diagram, dynamic operation and flow of events.
Finally, the theoretical foundations are outlined, providing the basic technical knowledge to undertake the project.
%\item In Chapter~\ref{ch:state-art}, the state of the art of remote controlled
%vehicles is presented.
%\item In Chapter~\ref{ch:theor-found} lays out the theoretical foundations for
%  project development,
%  namely the project development methodologies and associated tools, and the
%  communications technologies.
%\item In Chapter~\ref{ch:requirements-specs} are identified the key requirements
%  and constraints the system being developed must meet from the end-user
%  perspective (requirements) and, by defining well-established boundaries within
%  the project resources (time, budget, technologies and know-how), the list of
%  spefications is obtained.
%\item After defining the product specifications, the solutions space is explored
%  in Chapter~\ref{ch:analysis}, providing the rationale for viable solutions and
%  guiding the designer towards a best-compromise solution, yielding the
%  preliminary design and the foreseen tests to the specifications.
%\item The preliminary design is further refined in Chapter~\ref{ch:design} and
%  decomposed into tractable blocks (subsystems) which can be designed
%  independently and assigned to different design teams, allowing the transition
%  to the implementation phase.
%\item Next, in Chapter~\ref{ch:implementation}, the design solution is
%  implemented into the target platforms.
%\item Then, in Chapter~\ref{ch:testing}, the implementation is tested at the
%  subsystem level (unit testing) and system level (integrated testing),
%  analysing and comparing the attained performance with the expected one.
%\item After product testing, in Chapter~\ref{ch:verif-valid}, the specifications
%  must be verified and validated by an external agent.
%  subsystem level (unit testing) and system level (integrated testing),
%  analysing and comparing the attained performance with the expected one.
%\item Chapter~\ref{ch:conclusion} gives a summary of this report as well as
%  prospect for future work.
\item Lastly, the appendices (see Section~\ref{ch:Append}) contain detailed
  information about project planning and development.
\end{itemize}


%%% Local Variables:
%%% mode: latex
%%% TeX-master: "../../../dissertation"
%%% End:
