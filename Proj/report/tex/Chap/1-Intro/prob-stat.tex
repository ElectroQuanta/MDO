\section{Problem statement}%
\label{sec:prob-stat}
%The first step of the project is to clearly define the problem, as a result of
%the contract established between the client and the project team, yielding, in
%this case, the following project statement:
%
%``Design a remote control with three buttons that can
%remotely control the television (TV). It should be very
%light, powered by batteries and controls your TV via an
%infrared emitter. The TV has a built-in infrared receiver. A
%button on the remote control switches the TV on/off and
%will be labeled with the word "Power". The other two
%buttons are used to scroll up/down and select the available
%channels and they are labeled with the arrows up/down.''

%COVID pandemics presented a landmark on human interaction, greatly reducing the
%contact between people and surfaces. Thus, it is an imperative to provide people
%with contactless interfaces for everyday tasks. People redefined their
%purchasing behaviors, leading to a massive growth of the online
%shopping. However, some business sectors, like clothing or perfumes, cannot
%provide the same user experience when moving online.
%Therefore, one proposes to close that gap by providing a marketing digital
%outdoor for brands to advertise and gather customers with contactless
%interaction.
%
%Scenting marketing is a great approach to draw people into stores.
%Olfactory sense is the fastest way to the brain, thus, providing an exceptional
%opportunity for marketing~\cite{news-harvard} --- ``75\% of the emotions we generate on a daily basis are affected by smell. Next
%to sight, it is the most important sense we have''~\cite{lindstrom2006brand}.
%
%Combining that with additional stimuli, like sight and sound, can
%significantly boost the marketing outcome. Brands can buy advertisement space
%and time, selecting the videoclips to be displayed and the fragrance to be
%used at specific times, drawing the customers into their stores.
%
%Marketing also leverages from better user experience, thus, user interaction is
%a must-have, providing the opportunity to interact with the customer. In this
%sense, when users approach the outdoor a gesture-based interface will be
%provided for a brand immersive experience, where the user can take pictures or
%create GIFs with brand specific image filters and share them through their
%social media, with the opportunity to gain several benefits.

The first step of the project is to clearly define the problem, taking into
consideration the problem's context and motivation and exploiting the market opportunities.

\emph{The project consists of a \gls{mdo} with sound and
video display, and fragrance emission selected by the brands, providing a gesture-based interface for
user interaction to create pictures and \gls{gif}s, brand-specific, and share them on
social media.} It is comprised of several modes:
\begin{item-c}
\item \emph{normal mode (advertisement mode)}: the \gls{mdo} will provide
  sound, video and fragrance outputs.
\item \emph{interaction mode}: When a user approaches, the \gls{mdo} it will
go into interaction mode, turning on and displaying the camera feed and waiting
for recognizable gestures to provide additional functionalities, such as
brand-specific image filters.
\item \emph{sharing mode}: after a user take a picture or create a \gls{gif}, it
  can share it across social media.
\end{item-c}

Brands can buy advertisement space and time, selecting the videoclips to be
displayed and the fragrance to be used at specific times, drawing the customers
into their stores. Customers can be captivated by the combination of sensorial
stimuli, the gesture-based interaction, the immersive user experience provided
by the brands --- feeling they belong in a TV advertisement, and the opportunity
to gain several benefits, e.g., discount coupons.
%%% Local Variables:
%%% mode: latex
%%% TeX-master: "../../../dissertation"
%%% End:
