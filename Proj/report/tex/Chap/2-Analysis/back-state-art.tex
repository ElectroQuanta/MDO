\section{Background and State of the Art}
\label{sec:background-state-art}

%\section{Theoretical foundations}
%\label{sec:theor-found}
%The theoretical foundations provide the basic technical knowledge for project
%undertaking. In that sense, it is important to understand the principle of
%operation and the related technologies, namely the infrared communication
%protocol consisting of well-established data frames, specific to each
%manufacturer and at specific bandwith. It should be highlighted that the
%communication protocol information is critical for the correct behavior of the
%TV remote control, as the latter must stimulate the TV, complying to this protocol.
%
%Pushing a button on a remote control sets in motion a series of events
%that causes the controlled device to carry out a command. The process
%can be generally described as:
%\begin{enum-c}
%\item 
%pushing the button on the remote control causes a touch to the contact beneath it and complete the button circuit on the circuit board. The integrated circuit detects this.
%\item 
%The integrated circuit sends the binary of the button function command to the
%infrared \gls{led} at the front of the remote.
%\item 
%The \gls{led} emits a series of light pulses that corresponds to the binary the button command.
%\end{enum-c}
%
%As an example, one can take a look at the clicking on the ``volume up'' button
%on a Sony TV remote (Fig.~\ref{fig:btncode}):
%%
%  %\vspace{-5mm}
%%  
%%
%\begin{figure}[htb!]
%\centering
%    \includegraphics[width=0.45\columnwidth]{./img/buttoncode.png}
%  \caption{Example of wave generator for "volume up" from~\cite{btncode}}%
%\label{fig:btncode}
%\end{figure}
%
%The remote signal includes more than the command for ``volume up''. It sends
%several bits of information to the receiving device, establishing a
%communication protocol, including:
%\begin{item-c}
%\item a ``start'' command
%\item the command code for ``volume up''
%\item the device address (so the TV knows the data is intended for it)
%\item a ``stop'' command (triggered when you release the ``volume up'' button)
%\end{item-c}
%
%In this case, the buttons that are needed and its codes are:
%\begin{item-c}
%\item
%Power On = 001 0101
%\item
%Power Off = 010 1111
%\item
%Volume Up = 001 0010
%\item
%Volume Down = 001 0011
%\end{item-c}
%%
%  %\vspace{-5mm}
%%  
%\subsection{Reverse engineering}
%\label{sec:reverse-engineering}
%The contract established between the client (Samsung company) and the developer
%team (the authors) imposes the disclosement of the required information about
%the communication protocol. However, this is not always necessarily the case. As
%such, it is important to have a backup plan, which, in this case, corresponds to
%perform reverse engineering on the communication protocol.
%
%For this endeavour, an ``attack'' can be performed on the TV, by stimulating it
%at varying frequencies and set of commands and observing its effect. Obviously,
%the complexity grows with the number of required commands, but as in this case
%there are only 3 commands, this can be feasible. Furthermore, this can be
%bootstrapped by using available TV remote control emulators. An example setup
%can be connecting an \gls{ir} receiver at a Raspberry Pi, as illustrated in
%Fig.~\ref{fig:rasp-lirc} and loading a package, called \gls{lirc} that allows you to decode and send infrared signals of many (but not all) commonly used remote controls.
%
%The most important part of LIRC is the lircd daemon which decodes IR signals
%received by the device drivers and provides the information on a socket. It also
%accepts commands for IR signals to be sent if the hardware supports
%this~\cite{lirc}. Additionally, the sent IR signals can be used to identify the
%emitter characteristics, if already present in the database, or simply recorded
%for later usage. For the present use case, the list of available commands for
%Samsung TVs can also be obtained for the database for actual TV ``attack''.
%%
%  %\vspace{-5mm}
%%  
%%
%\begin{figure}[htb!]
%\centering
%    \includegraphics[width=0.43\columnwidth]{./img/rasp-reverse-engineering-setup.jpg}
%  \caption{Example setup for reverse engineering of TV remote control commands
%    using an emulator~\cite{rasp-lirc}}%
%\label{fig:rasp-lirc}
%\end{figure}

%%% Local Variables:
%%% mode: latex
%%% TeX-master: "../../../dissertation"
%%% End:
