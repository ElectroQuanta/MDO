\chapter{Analysis}
\label{ch:analysis}
In the analysis phase, the product requirements are derived --- defining the client expectations
for the product --- as well as the project constraints --- what the environments
limits about the product. Based on the set of requirements and constraints, a
system overview is produced, capturing the main features and interactions with
the system, as well as its key components.

Finally, the theoretical foundations are outlined,
providing the basic technical knowledge to undertake the project.

% Requirements and constraints
\section{Requirements and Constraints}
\label{sec:req-const}

The development requirements are divided into functional and non-functional if they pertain to main functionality or secondary one, respectively. Additionally, the constraints of the project are classified as technical or non-technical.

\subsection{Functional requirements}
\label{sec:funct-requ}

\begin{item-c}
\item Take pictures and~\gls{gif}s;
\item Show advertisements through a screen and speakers;
\item Have fragrance emission;
\item Have contactless feedback through gestures recognition.
\end{item-c}
\subsection{Non-functional requirements}
\label{sec:non-funct-requ}

\begin{item-c}
\item Low power consumption;
\item Have a comfortable and non evasive interface (user-friendly);
\item Have low latency between local and remote server
\end{item-c}
\subsection{Technical constraints}
\label{sec:techn-constr}

\begin{item-c}
\item Use device drivers;
\item Use Makefiles;
\item Use C/C++;
\item Make a~\gls{cps};
\item Use Raspberry Pi as the development board;
\item Use compatible~\gls{hw} with the development board;
\item Use buildroot;
\item Work with Linux.
\end{item-c}
\subsection{Non-technical constraints}
\label{sec:non-techn-constr}

\begin{item-c}
\item Project deadline at the end of the semester;
\item Pair work flow;
\item Limited budget.
\end{item-c}

%The requirements defined the client expectations for the TV remote control,
%namely:
%\begin{item-c}
%\item Remotely operated
%\item Low weight
%\item Powered by batteries
%\item 3 buttons: Power (Off/On); Up and Down for channel selection.
%\item Infrared emitter response time (system output response time): 100 ms
%\item The TV remote may be upgraded in the future to use more buttons
%\end{item-c}
%%
%  %\vspace{-5mm}
%%  
%\section{Constraints}
%\label{sec:constraints}
%The project constraints are the limitations the environment imposes on it, namely:
%\begin{item-c}
%\item the TV remote must contain an infrared emitter (the TV already has an infrared receiver)
%\item The TV remote control must supply the required data frames imposed by the TV
%  manufacturer
%\item Data frames may not be provided by the client
%\item Security concerns are defined by the data frames and the specific
%  communication frequency imposed by the TV manufacturer
%\item 1 week deadline: 14 h
%\item Manpower: 2 people
%\item Budget:
%  \begin{itemize}
%  \item HW (parts acquisition and assembly): fixed costs --- 1 EUR/unit (1000
%    batch production)
%    \begin{itemize}
%    \item TV remote Shell
%    \item TV remote membrane
%    \item Data acquisition \& Infrared emitter PCB
%    \end{itemize}
%  \item Development: project --- 20 EUR per hour per person, totalling 560 EUR +
%    IVA
%  \end{itemize}
%\end{item-c}
%
  %\vspace{-5mm}
%  

%%% Local Variables:
%%% mode: latex
%%% TeX-master: "../../../dissertation"
%%% End:

% System overview
%
\section{System overview}
\label{sec:system-overview}
The system overview presents a global view of the system, considering its main
features, components and interactions. It is not intended to be complete, but
rather provide a basis for the outline of the system architecture.
Fig.~\ref{fig:sys-overview} presents the \gls{mdo} system overview.
%
\begin{figure}[htb!]
\centering
    \includegraphics[width=1.0\columnwidth]{./img/sys-overview.png}
  \caption{\gls{mdo} system overview}%
\label{fig:sys-overview}
\end{figure}

Considering the system interactions, three main actors were identified:
\begin{enum-c}
\item \emph{Brand}: represents the brands contracting the advertisement
  services;
\item \emph{Company staff}: the development company staff, which can monitor and
  control the outdoor.
\item \emph{User}: the user (the target audience of the advertisiment)
  interacting with the system.
\end{enum-c}

Considering the data flow across the \textbf{MDO system}, three main subsystems were
identified: \textbf{\gls{mdo-rc}}, \textbf{\gls{mdo-rs}}, and
\textbf{\gls{mdo-l}}. The rational behind this initial decomposition is
explained next.

\subsection{MDO Remote Client}
The \emph{Brand} and \emph{Company Staff} members require a remote \gls{ui} (front-end) to
interact with the system: the former to configure the advertisements being
displayed at the \gls{mdo} and purchase them; the latter to remotely monitor and
control the operation of the \gls{mdo}. Thus, it is clear that \emph{an
  authentication mechanism must be provided for the remote \gls{ui}}.

The data is then dispatched to the back-end, where it is processed and feed back
to the \gls{ui} user and/or sent to the remote server, via \gls{tcp-ip}
comprising the data logic component of the \gls{ui}.
%
%
\subsection{MDO Remote Server}
\label{sec:mdo-remote-server}
Although the \gls{mdo-rc} could communicate directly with the \gls{mdo-l}, this
is not desirable or a good architecture mainly due to: communications failure could
result in data loss, compromising the system's integrity; the remote client and
the local system become tightly coupled, meaning the remote client must be aware
of all the available local systems; if the data storage in the local system
fails, the remote client would have to provide the backup information.

Thus, a remote server component is included, providing the access and management
of the system databases, pertaining to the \emph{Brand}, \emph{Company}, and
\emph{MDO Local system}. The first two provide the historical information of the
\texttt{Brand} and \texttt{Company} entities, and the last one the information
related to all of the \texttt{\gls{mdo-l}} systems in operation.

The main functions of the \texttt{\gls{mdo-rs}} are:
\begin{item-c}
\item \emph{UI requests responses}: when a \gls{ui} user requests/modifies
  some information from the database, the server must provide/update it.
\item \emph{\gls{mdo-l} monitoring and control}: provide command dispatch and
  feedback to the \texttt{Company} staff for remote monitoring and control of
  the device.
\item \emph{\gls{mdo-l} update}: periodically check for start times of each
  \gls{mdo-l} device and transfer the relevant data to it.
\end{item-c}
%

\subsection{MDO Local system}
\label{sec:mdo-local-system}




%%% Local Variables:
%%% mode: latex
%%% TeX-master: "../../../dissertation"
%%% End:

% System Architecture

\section{System architecture}
\label{sec:system-architecture}

\subsection{Hardware architecture}
\label{sec:hardw-arch}
%
The diagram in Fig.\ref{fig:hw-arch} represents an initial hardware big picture in order to facilitate the objective identification.
As it can bee seen, the diagram is divided in four distinguished parts: \emph{External Environment}, \emph{Local System}, \emph{Remote Server} and \emph{Remote Client}.

Firstly, the \texttt{External Environment} represents all the environment that interacts with the system. In this case, these are all its users - normal users, brands and staff.

Secondly, the \texttt{Local System} is composed for the main controller, which is the Raspberry Pi 4B. 
This \gls{mcu} is responsible to controll all the Local System and to establish connection with the remote server through its included WiFi module. 
The board is powered connecting it to the electrical network. 
Then, it has several blocks connected to it:
%
\begin{item-c}
\item \emph{Motion Detection}: Used to detect the users and switch from normal mode to interaction mode;
\item \emph{Fragrance Diffusion Actuator}: used to diffuse the fragrance onto the air;
\item \emph{Camera}: Used to capture image that is then processed;
\item \emph{Speakers}: Used to produce advertisements sounds;
\item \emph{Screen}: Used to produce video clips of advertisements.
\end{item-c}
%

In third place, the \texttt{Remote Server} has a server node running in another machine that can be one computer or a main frame.
The remote server establishes connection with the cloud that has stored all the data from all databases.

Lastly, the \texttt{Remote Client} which can be a computer, a tablet or a smart phone to run the \gls{mdo} management application.
%
\begin{figure}
\centering
    \includegraphics[width=0.9\columnwidth]{./img/HW_Architecture.png}
  \caption{~\gls{hw} Architecture Diagram}%
\label{fig:hw-arch}
\end{figure}
%
%
\subsection{Software architecture}
\label{sec:softw-arch}



%%% Local Variables:
%%% mode: latex
%%% TeX-master: "../../../dissertation"
%%% End:

% Subsystem decomposition

\section{Subsystem decomposition}
\label{sec:subsyst-decomp}

For each subsystem, do:
\begin{enum-c}
\item User mockups
\item Events
\item Use cases
\item State machine diagram
\item Sequence diagram
\end{enum-c}

%
\subsection{Remote Client}
\label{sec:remote-cli-decomp}
%
In this section the remote client is analyzed, considering its events, use cases, dynamic operation and the flow of events.

\subsubsection{User mockups}
\label{sec:user-mockups-1}
%
In Fig.~\ref{fig:user-mockups-rc} is illustrated the user mockups for the remote client. 
It intends to clarify how does the~\gls{ui} works for the two different sides: Brands and Company (staff).

The initial state of the~\gls{mdo-rc}'s~\gls{ui} is depicted in thick border outline: the 'Sign In' window. 
If the \texttt{User} makes a mistake in its username and/or password, it will be shown an error message. 
Also, the 'Sign In' window has an option to recover the password, which sends an e-mail to switch password.
If the \texttt{User} still remembers its credentials, the app flows through one out of two possibilities: if the user is an admin, goes to the admin main menu, or else if the user is a brand, it will appear the brand main menu.

Firstly, the \texttt{Admin} workflow:
%
\begin{itemize}
\item The \texttt{Admin} main menu contains a drop down button with all the stations available. Choosing one of them, the \texttt{Admin} can turn it On/Off, see it's current mode and the current brand ad being displayed. Also, the \texttt{Admin}  can log out and choose between two different paths:
%
\begin{itemize}
\item \emph{Statistics}: It is possible to see various statistics of all different brands that are currently playing on the station: the number of times that the ad was shown, the number of pictures/\gls{gif}s and shared posts, the fragrance slot and quantity (percentage) and the days remaining for the rent to end.
It is also possible to deactivate the add if something wrong occurs and go back to the previous menu.
\item \emph{Users}: In this window, the admin can manage all users and see their information.
It is possible to the admin to change the type of user to brand or to admin, and it can also remove its type.
Also, the admin can delete users from the database. 
\item \emph{Ads to Activate}: In this window, the \texttt{Admin} can handle all the ads that the brands are intending to rent.
For that, the \texttt{Admin} needs to see if everything is in order, such as if all the videos that the brand wants to display are in order (in case there are some  or decontextualized videos), if it has a filter, a fragrance and a slot.
After that, the \texttt{Admin} can either accept or deny the ad.
If it accepts the ad, it is shown a success message and the ad is added to the station with its preferences.
If the ad is denied, the \texttt{Admin} needs to send a reason why it denied the ad, that is consequently send to the brand's email.
\end{itemize}
%
\end{itemize}

Secondly, the \texttt{Brand} workflow:
\begin{itemize}
\item The \texttt{Brand} main menu contains a welcome message, a notification bell to see if another ad was accepted or denied and three buttons - Rented, To Rent and Log Out.
The 'Log Out' button logs the \texttt{Brand} out of its account, the other two buttons switch to different widgets:
%
\begin{itemize}
\item \emph{Rented}: The \texttt{Brand} can see all statistics of all its rented ads on different stations that it rented.
That statistics are: status, number of times the ad was shown, the fragrance slot and quantity (percentage), the number of pictures/\gls{gif}s taken, the number of shared posts and the number of days remaining to end its rent.
The 'Go Back' button goes back to the previous menu.
\item \emph{To Rent}: The \texttt{Brand} can rent ads in the same station or in other stations.
To that happens, it is only needed to choose the target hours and then a calendar will show what days are available to that hours, then after choosing the days, the \texttt{Brand} need to upload a filter and a .zip file with a maximum of ten videos. 
Finally, the \texttt{Brand} needs to select the fragrance that wants to spill to the air and select 'Rent'. After that, a success message will be shown and the ad will enter in wait list for an \texttt{Admin} check if everything is in order.
\end{itemize}
%
\end{itemize}

It is also possible to register a new user through the 'Register' button.
This opens a window to type a username, a password, confirm the password and the e-mail.
If everything is in order, the user is created with the default user type of Brand.

Finally, at any time, it can occur the loss of internet connection, which toggles an error message informing the automatic log out of the account.
\begin{figure}[htb!]
\centering
    \includegraphics[width=0.9\columnwidth]{./img/user-mockups-rc.png}
  \caption{User mockups: remote client}%
\label{fig:user-mockups-rc}
\end{figure}

\subsubsection{Events}
\label{sec:events-1}
Table~\ref{tab:events-rc} presents the most relevant events for the Local system, categorizing them by their source and
synchronism and linking it to the system’s intended response.

%
\begingroup
\renewcommand{\arraystretch}{0.7} % Default value: 1
% Please add the following required packages to your document preamble:
% \usepackage{graphicx}
\begin{table}[]
\centering
\caption{Events: remote client}
\label{tab:events-rc}
\resizebox{\textwidth}{!}{%
\begin{tabular}{llll}
\hline
\textbf{Event} &
  \textbf{System response} &
  \textbf{Source} &
  \textbf{Type} \\ \hline
Login &
  \begin{tabular}[c]{@{}l@{}}The system verifies if the user credentials\\ are correct and what type of user is and\\ asks for data from databases\end{tabular} &
  User &
  Asynchronous \\ \hline
Verify internet connection &
  Periodically verify internet connection &
  Remote Client &
  Synchronous \\ \hline
Statistics &
  \begin{tabular}[c]{@{}l@{}}Request to the Remote Server all the\\  information to show statistics from \\ all stations and brands\end{tabular} &
  User (Admin) &
  Asynchronous \\ \hline
Accept/Deny ad &
  \begin{tabular}[c]{@{}l@{}}Send information to the Remote \\ Server if the ad is either accepted\\ or denied and if so, why\end{tabular} &
  User (Admin) &
  Asynchronous \\ \hline
Power On/Off Station &
  \begin{tabular}[c]{@{}l@{}}Send command to Remote Server\\  to Power On/Off a certain station\end{tabular} &
  User (Admin) &
  Asynchronous \\ \hline
Rented &
  \begin{tabular}[c]{@{}l@{}}Request to the Remote Server all the\\  information to show statistics from \\ all stations the brand rented\end{tabular} &
  User (Brand) &
  Asynchronous \\ \hline
Rent &
  \begin{tabular}[c]{@{}l@{}}Send to the Remote Server all the \\ information of rent from the brand, \\ all the videos and the filter\end{tabular} &
  User (Brand) &
  Asynchronous \\ \hline
Forgot Password &
  \begin{tabular}[c]{@{}l@{}}Send e-mail to the user that has \\ forgotten his password\end{tabular} &
  User &
  Asynchronous \\ \hline
\end{tabular}%
}
\end{table}
%
%
%
\subsubsection{Use cases}
\label{sec:use-cases-1}
%
Fig.~\ref{fig:use-cases-rc} depicts the use cases diagram for the \texttt{Remote Client}, describing how the system should respond under various conditions to a request from one of the stakeholders to deliver a specific
goal.

The \texttt{Admin} and the \texttt{Brand} interact with the \texttt{Remote Client} and this last interacts with the \texttt{Remote Server} to process commands, such as query databases or power on/off machines.

The \texttt{Admin} can Manage the Station, which includes Power On/Off Station, Manage Ads to Activate and Enable/Disable an Ad.
It can also manage users, removing or modifying them. All these use cases are processed from the \texttt{Remote Client} and are requested to the \texttt{Remote Server}.

The \texttt{Brand} can see Rented Ads, Rent Ads, See notifications and register. All these cases are also processed from the \texttt{Remote Client} and are requested to the \texttt{Remote Server}.

There are some use cases that are common to the  \texttt{Admin} and to the \texttt{Brand}: Login and Logout.


\begin{figure}[htb!]
\centering
    \includegraphics[width=0.9\columnwidth]{./img/use-cases-rc.png}
  \caption{Use cases: remote client}%
\label{fig:user-cases-rc}
\end{figure}


\subsubsection{Dynamic operation}
\label{sec:dyn-oper-1}
State machine diagram

\subsubsection{Flow of events}
\label{sec:flow-events-1}
Sequence diagram

%%% Local Variables:
%%% mode: latex
%%% TeX-master: "../../../dissertation"
%%% End:

%
\subsection{Remote server}
\label{sec:remote-serv-decomp}

\subsubsection{User mockups}
\label{sec:user-mockups-2}

\subsubsection{Events}
\label{sec:events-2}

\subsubsection{Use cases}
\label{sec:use-cases-2}

\subsubsection{Dynamic operation}
\label{sec:dyn-oper-2}
State machine diagram

\subsubsection{Flow of events}
\label{sec:flow-events-2}
Sequence diagram

%%% Local Variables:
%%% mode: latex
%%% TeX-master: "../../../dissertation"
%%% End:

%
\subsection{Local system}
\label{sec:local-system}

\subsubsection{Events}
\label{sec:events}

\subsubsection{Use cases}
\label{sec:use-cases}

\subsubsection{Dynamic operation}
\label{sec:dyn-oper}
State machine diagram

\subsubsection{Flow of events}
\label{sec:flow-events}
Sequence diagram

%%% Local Variables:
%%% mode: latex
%%% TeX-master: "../../../dissertation"
%%% End:

%%% Local Variables:
%%% mode: latex
%%% TeX-master: "../../../dissertation"
%%% End:

% Project planning and budget
% \section{Project planning}
\label{sec:project-planning}

\subsection{Budget}
\label{sec:budget}


%%% Local Variables:
%%% mode: latex
%%% TeX-master: "../../../dissertation"
%%% End:

% Theoretical foundations
% \input{./tex/Chap/2-Analysis/theor-found}
%
%%% Local Variables:
%%% mode: latex
%%% TeX-master: "../../../dissertation"
%%% End:
