%
\section{System architecture}
\label{sec:system-architecture}
In this section, the system architecture is devised in the \gls{hw} and \gls{sw} components, using the system overview as a starting point. 

\subsection{Hardware architecture}
\label{sec:hardw-arch}
%
Fig.~\ref{fig:hw-arch} illustrates an initial hardware big picture that fulfils
the system's goals, meeting its requirements and constraints.
As it can bee seen, the diagram is divided in four distinct parts:
\texttt{External Environment}, \texttt{Local System}, \texttt{Remote Server} and
\texttt{Remote Client}.
%
\begin{figure}[htb!]
\centering
    \includegraphics[width=0.8\columnwidth]{./img/hw-arch.png}
  \caption{\gls{hw} architecture diagram}%
\label{fig:hw-arch}
\end{figure}
%

Firstly, the \texttt{External Environment} represents all the environment that
interacts with the system. In this case, these are all its users --- normal
users, brands and company staff (Administrator role).

Secondly, the \texttt{Local System} is composed of the main controller, which is the Raspberry Pi 4B. 
This \gls{soc} is responsible to control all the \texttt{Local System} and to establish a
connection with the \texttt{Remote Server} through its included WiFi
module. Additionally, the \texttt{Local system} also communicates with the
\textit{Cloud} to share contents on social media, and, potentially, to access
image filtering \gls{api}s.
The \texttt{Local System} is powered through a \gls{ac}-\gls{dc} power
converter, and, potentially, a step-down converter --- \texttt{Power Supply Module}.
The main board has several blocks connected to it:
%
\begin{item-c}
\item \emph{Motion Detection}: used to detect the users and switch from normal mode to interaction mode;
\item \emph{Fragrance Diffusion Actuator}: used to diffuse the fragrance into the air;
\item \emph{Camera}: used to capture image that is then processed;
\item \emph{Speakers}: used to reproduce advertisements sounds;
\item \emph{Screen}: used to display video clips of advertisements.
\end{item-c}
%

In third place, the \texttt{Remote Server} has a server node running in another machine that can be one computer or a main frame.
The remote server stores all databases which the \texttt{Remote Client} and
\texttt{Local System} may need to access and serves as a proxy server to enable
the \texttt{Admin} users to control and monitor the \texttt{Local System}.

Lastly, the \texttt{Remote Client} runs the \gls{mdo} management application,
which can be deploy to a computer (like the Raspberry Pi), a tablet or a smartphone.
%
\subsection{Software architecture}
\label{sec:softw-arch}
In this section the \gls{sw} architecture for \gls{mdo-rc}, \gls{mdo-rs}, and
\gls{mdo-l} subsystems is presented, defining its \gls{sw} stack.

\subsubsection{MDO remote client}
\label{sec:mdo-remote-client}
%
Fig.~\ref{fig:sw-arch-rc} illustrates the \gls{sw} architecture for the remote
client, representing its \gls{sw} stack.
It is comprised of the following layers:
\begin{item-c}
\item \emph{Application}: contains the remote client application. The
  \texttt{Brand} and \texttt{Admin} members interact with the \gls{ui}, which is
  the visual part of the interface. The \texttt{\gls{ui} engine} is notified and
  handles all \gls{ui} events --- internal or external --- providing the \texttt{UI}
  with feedback for its users. The relevant commands
  are then parsed --- \texttt{Parser} component --- and responded. The commands
  are then translated to the appropriate \gls{db} queries and responded through
  the \texttt{DB Manager}. The \texttt{Comm Manager} is responsible for
  encapsulating the \gls{db} queries into the respective \gls{tcp-ip} frames to
  be sent to the \texttt{Remote Server} as well as unwrap the incoming server
  responses.
\item \emph{Middleware}: contains the \gls{tcp-ip} framework supporting these
  communication protocols as part of \gls{osi} model for internet
  applications. It manages the incoming/outgoing \gls{tcp-ip} frames by
  providing the adequate protocol handshaking and queueing and timing aspects of
  the bytes to send/receive.
\item \emph{OS \& BSP} --- \gls{os} \& \gls{bsp}: it contains the low-level and
  communication drivers required to handle input (keyboard/touch), output
  (screen) and communication to the \texttt{Remote Server}.
\end{item-c}
It should be noted that for desktop and mobile applications, the
\texttt{Middleware} and \texttt{OS \& BSP} layers are usually abstracted by the
\gls{os}, thus, the relevant \gls{api}s should be used.
%
\begin{figure}[htb!]
\centering
    \includegraphics[width=0.55\columnwidth]{./img/sw-arch-rc.png}
  \caption{\gls{sw} architecture diagram: remote client}%
\label{fig:sw-arch-rc}
\end{figure}

\subsubsection{MDO remote server}
\label{sec:mdo-remote-server-1}
%
Fig.~\ref{fig:sw-arch-rs} illustrates the \gls{sw} stack for architecture for
the remote server.
It is comprised of the following layers:
\begin{item-c}
\item \emph{Application}: contains the remote server application. It provides a
  \gls{cli} to handle \texttt{Remote client} requests.  The \gls{cli} engine
  is notified and handles all \gls{ui} events --- internal or external ---
  providing the appropriate feedback. The relevant commands
  are then parsed --- \texttt{Parser} component --- and responded: \gls{db}
  queries are handled by the \texttt{\gls{rdbms}} issuing \gls{db} transactions;
  other commands received from the \texttt{Remote Client} are handled internally
  and translated, being dispatched to the \texttt{Local
    System} by the \texttt{Comm Manager}  (via \texttt{Communication drivers}). Internal events can also
  trigger the \texttt{\gls{rdbms}} to issue database transactions for the
  \texttt{Remote Client} or \texttt{Local System}.
  The \texttt{Comm Manager} is responsible for wrapping\slash unwrapping the data
  frames received by or sent to the \texttt{Remote Client} or \texttt{Local System}.
\item \emph{Middleware}: contains the \gls{rdbms} framework supporting the
  management of the relational databases using database transactions.
\item \emph{OS \& BSP} --- \gls{os} \& \gls{bsp}: it contains the \texttt{Communication}
  drivers to the handle requests from the \texttt{Remote Client}, and the
  \texttt{File I/O} drivers to manipulate \gls{db} transactions from\slash to storage.
\end{item-c}
%
\begin{figure}[htb!]
\centering
    \includegraphics[width=0.55\columnwidth]{./img/sw-arch-rs.png}
  \caption{\gls{sw} architecture diagram: remote server}%
\label{fig:sw-arch-rs}
\end{figure}
%
It should be noted that the \texttt{Remote Server} main functions are:
\begin{item-c}
\item provide relational databases for easier management of all entities and
  respective data in the system;
\item decompose the relationship many-to-many, between the remote clients and
  local systems --- many remote clients may want to connect to different local
  systems; 
\item decouple the architecture as the \texttt{Remote Client} should not know
the \gls{ip} address of every local system it may potentially try to access,
acting as a proxy server.
\end{item-c}
%
\subsubsection{MDO local system}
\label{sec:mdo-local-system-1}
%
Fig.~\ref{fig:sw-arch-local} illustrates the \gls{sw} stack for architecture for
the \texttt{Local System}.
It is comprised of the following layers:
\begin{item-c}
\item \emph{Application}: contains the local system application. It provides a
  \gls{ui} to handle \texttt{User} interaction.  The \texttt{Interface} engine
  is notified and handles all \gls{ui} events --- internal or external ---
  through gesture recognition, providing the appropriate feedback. The relevant
  commands are then parsed --- \texttt{Supervisor} component --- and responded: \gls{db}
  queries are handled by the \texttt{Database manager} issuing \gls{db}
  transactions for internal databases;
  commands received from the \texttt{Remote Server} to monitor or control the
  system are handled internally
  and responded back by the \texttt{Comm Manager}  (via \texttt{Communication
    drivers}); mode management is performed.
  Internal events can also trigger the \texttt{Database manager} to issue
  database transactions to update the \texttt{Local System}.
  The \texttt{Comm Manager} is responsible for wrapping\slash unwrapping the data
  frames received by or sent to the \texttt{Remote Server}.
\item \emph{Middleware}: contains: the \gls{db} framework supporting the
  management of the internal databases using database transactions; the \gls{cv}
  framework that handles gesture and facial recognition; image filtering and
  \gls{gif} frameworks for multimedia; social media framework.
\item \emph{OS \& BSP} --- \gls{os} \& \gls{bsp}: contains: the \texttt{Communication}
  drivers to the handle requests from the \texttt{Remote Server} and for social
  media sharing, and, potentially the \gls{api} calls to cloud-based image
  filtering frameworks, depending on the application profiling; the
  \texttt{File I/O} drivers to manipulate internal \gls{db} transactions
  from\slash to storage; audio, video and fragrance diffuser actuator drivers
  for normal mode; the camera driver for camera feed; the detection sensor
  driver to signal a \texttt{User} is in range, triggering the switch from
  normal mode to interaction mode.
\end{item-c}
The \texttt{Local system} is a \texttt{soft real-time} system, as no mandatory
deadlines must be met.
%
\begin{figure}[htb!]
\centering
    \includegraphics[width=1.0\columnwidth]{./img/sw-arch-local.png}
  \caption{\gls{sw} architecture diagram: local system}%
\label{fig:sw-arch-local}
\end{figure}
%
%
%%% Local Variables:
%%% mode: latex
%%% TeX-master: "../../../dissertation"
%%% End:
