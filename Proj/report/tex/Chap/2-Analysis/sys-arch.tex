
\section{System architecture}
\label{sec:system-architecture}

\subsection{Hardware architecture}
\label{sec:hardw-arch}
%
The diagram in Fig.\ref{fig:hw-arch} represents an initial hardware big picture in order to facilitate the objective identification.
As it can bee seen, the diagram is divided in four distinguished parts: \emph{External Environment}, \emph{Local System}, \emph{Remote Server} and \emph{Remote Client}.

Firstly, the \texttt{External Environment} represents all the environment that interacts with the system. In this case, these are all its users - normal users, brands and staff.

Secondly, the \texttt{Local System} is composed for the main controller, which is the Raspberry Pi 4B. 
This \gls{mcu} is responsible to controll all the Local System and to establish connection with the remote server through its included WiFi module. 
The board is powered connecting it to the electrical network. 
Then, it has several blocks connected to it:
%
\begin{item-c}
\item \emph{Motion Detection}: Used to detect the users and switch from normal mode to interaction mode;
\item \emph{Fragrance Diffusion Actuator}: used to diffuse the fragrance onto the air;
\item \emph{Camera}: Used to capture image that is then processed;
\item \emph{Speakers}: Used to produce advertisements sounds;
\item \emph{Screen}: Used to produce video clips of advertisements.
\end{item-c}
%

In third place, the \texttt{Remote Server} has a server node running in another machine that can be one computer or a main frame.
The remote server establishes connection with the cloud that has stored all the data from all databases.

Lastly, the \texttt{Remote Client} which can be a computer, a tablet or a smart phone to run the \gls{mdo} management application.
%
\begin{figure}
\centering
    \includegraphics[width=0.9\columnwidth]{./img/HW_Architecture.png}
  \caption{~\gls{hw} Architecture Diagram}%
\label{fig:hw-arch}
\end{figure}
%
%
\subsection{Software architecture}
\label{sec:softw-arch}



%%% Local Variables:
%%% mode: latex
%%% TeX-master: "../../../dissertation"
%%% End:
