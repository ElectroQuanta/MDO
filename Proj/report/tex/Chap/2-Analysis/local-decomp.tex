%
\subsection{Local system}
\label{sec:local-system}
In this section the local system is analyzed, considering its events, use cases,
dynamic operation and the flow of events.
%
\subsubsection{User mockups}
\label{sec:user-mockups}
Fig.~\ref{fig:user-mockups-local} illustrates the user mockups for the local
system. It intends to mimic the user interaction with the local system,
clarifying the user actions (gestures) and the respectives responses, as well as
the workflow, comprising its four modes.
%
\begin{figure}[htb!]
\centering
    \includegraphics[width=1.0\columnwidth]{./img/user-mockups-local.png}
  \caption{User mockups: local system}%
\label{fig:user-mockups-local}
\end{figure}

The initial state of the \gls{mdo-l}'s~\gls{ui} is depicted in thick border
outline, after a \texttt{User} has been detected --- \texttt{Interaction
  mode}. On the left it is the camera feed and on the right the commands ribbon,
containing the hints to use the system and the available options. As it can
been, the \texttt{User} can choose an option by hovering with pointing finger
over the desired option for a designated amount of time (e.g., 3 seconds).

The workflow can be as follows:
\begin{item-c}
\item If the \texttt{User} selects the \texttt{Image filter} option, the
  \texttt{Image filtering} view is shown, presenting the options to select
  filters (which can be scrolled through palm raising/lowering), to cancel or
  accept the image filter. If a filter is selected \texttt{filter1\_pressed}, it
  is applied, and if accepted it will return to \texttt{Interaction mode},
  keeping the filter on.
\item If the \texttt{User} selects the \texttt{Take Pic} option, \texttt{Picture
  mode} is started with a timer to allow the \texttt{User} to get ready before
actually taking the picture. The \texttt{User} can \texttt{Cancel} --- returning
to main menu --- or \texttt{Share} --- starting \texttt{Sharing mode}.
\item If the \texttt{User} selects the \texttt{Create GIF} option, \texttt{GIF
    mode (setup)} is started with a timer to allow the \texttt{User} to get
  ready before actually creating the \gls{gif}. After the \texttt{setup\_timer}
  is elapsed, the \texttt{GIF mode (operation)} starts, displaying a dial with
  the \gls{gif} duration until being complete. When the \texttt{gif\_timer}
  elapses, the \gls{gif} is created, enabling the \texttt{User} to
  \texttt{Cancel} --- returning to main menu --- or to \texttt{Share} ---
  starting \texttt{Sharing mode}.
\item Lastly, in the \texttt{Sharing mode}, the \texttt{User} can
  \texttt{Cancel} --- returning to main menu --- or select the
  social media network. After selecting the social media, the \texttt{User} can
  edit the post by entering its customized message and, if \texttt{Share} is
  pressed, a message box will appear displaying the status of the post sharing
  --- \texttt{Success} or \texttt{Error}.
\end{item-c}
%
\subsubsection{Events}%
\label{sec:events}
Table~\ref{tab:events-local} presents the most relevant events for the
\texttt{Local system}, categorizing them by their source and synchronism and
linking it to the system's intended response. A further division is done
separating \emph{\gls{ui} events} from the remaining ones.

\begingroup
\renewcommand{\arraystretch}{0.7} % Default value: 1
% Please add the following required packages to your document preamble:
% \usepackage{booktabs}
\begin{table}[hbt!]
\centering
\caption{Events: local system}
\label{tab:events-local}
\begin{tabular}{@{}llll@{}}
\toprule
\textbf{Event}                         & \textbf{System response}                                                                              & \textbf{Source}   & \textbf{Type} \\ \midrule
Power on                               & \begin{tabular}[c]{@{}l@{}}Initialize sensors and go \\ to Normal mode\end{tabular}                   & System maintainer & Asynchronous  \\ \midrule
User detected                          & \begin{tabular}[c]{@{}l@{}}Turn on camera feed and \\ go to Interaction mode\end{tabular}             & User              & Asynchronous  \\ \midrule
Command received                       & Parse it and respond                                                                                  & Remote Server     & Asynchronous  \\ \midrule
Database update                        & \begin{tabular}[c]{@{}l@{}}Request update of internal \\ databases to Remote Server\end{tabular}      & Database manager  & Asynchronous  \\ \midrule
Enable fragrance diffuser              & \begin{tabular}[c]{@{}l@{}}Enable fragrance diffusion for \\ a predefined period of time\end{tabular} & Local System      & Synchronous   \\ \midrule
Video ended                            & \begin{tabular}[c]{@{}l@{}}Playback the next video \\ on the queue\end{tabular}                       & Local System      & Synchronous   \\ \midrule
Check WiFi connection                  & Periodically check WiFi connection                                                                    & Local System      & Synchronous   \\ \midrule
\multicolumn{1}{c}{\textbf{UI events}} &                                                                                                       &                   &               \\ \midrule
Option selected                        & \begin{tabular}[c]{@{}l@{}}Track the option selected and \\ inform the UI engine\end{tabular}         & User              & Asynchronous  \\ \midrule
Image filter pressed                   & Go to Image filter view                                                                               & User              & Asynchronous  \\ \midrule
Filter selected                        & Detect User's face and apply filter                                                                   & User              & Asynchronous  \\ \midrule
Pic pressed                            & Go to Picture mode                                                                                    & User              & Asynchronous  \\ \midrule
Pic setup elapsed                      & Take picture                                                                                          & Local System      & Synchronous   \\ \midrule
GIF pressed                            & Go to GIF mode                                                                                        & User              & Asynchronous  \\ \midrule
GIF setup elapsed                      & Go to GIF operation                                                                                   & Local System      & Synchronous   \\ \midrule
GIF operation elapsed                  & Finish GIF                                                                                            & Local System      & Synchronous   \\ \midrule
Share mode pressed                     & Go to Sharing mode (selection)                                                                        & User              & Asynchronous  \\ \midrule
Keyboard pressed                       & Give feedback to user                                                                                 & User              & Asynchronous  \\ \midrule
Share post pressed                     & Upload post to designated social media                                                                & User              & Asynchronous  \\ \midrule
Share post status                      & Inform user about shared post status                                                                  & Cloud             & Asynchronous  \\ \bottomrule
\end{tabular}
\end{table}
\endgroup
%
\subsubsection{Use cases}
\label{sec:use-cases}
Fig.~\ref{fig:use-cases-local} depicts the use cases diagram for the
\texttt{Local System}, describing how the system should respond under various
conditions to a request from one of the stakeholders to deliver a specific
goal.

The \texttt{Admin} interacts with the \texttt{Remote Client} (through its
\gls{ui}) requesting the \texttt{Remote Server} to process commands, getting the
state of the device, adding a video or selecting the fragrance. Additionally,
the \texttt{Admin} may test the operation of the device: play video, test audio,
nebulize fragrance or test the camera. This last one tests the main
functionalities the \texttt{User} also utilizes, namely: select image filter,
apply/reject image filter, take picture, create \gls{gif} or share multimedia on
the social media.

A precondition for the interaction of the \texttt{Admin} with the \texttt{Local System} is the establishment of a remote
connection between the \texttt{Remote Server} and the \texttt{Local system},
verifying its credentials. However, there is another important use case for this
remote connection: the update of the \texttt{Local System}'s internal databases
from the \texttt{Remote server} which will sent appropriate commands for this purpose.
%
\begin{figure}[htb!]
\centering
    \includegraphics[width=0.85\columnwidth]{./img/use-cases-local.png}
  \caption{Use cases diagram: local system}%
\label{fig:use-cases-local}
\end{figure}

\subsubsection{Dynamic operation}
\label{sec:dyn-oper}
State machine diagram

\subsubsection{Flow of events}
\label{sec:flow-events}
Sequence diagram

%%% Local Variables:
%%% mode: latex
%%% TeX-master: "../../../dissertation"
%%% End:
