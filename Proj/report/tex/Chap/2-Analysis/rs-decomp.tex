%
\subsection{Remote server}
\label{sec:remote-serv-decomp}
In this section the remote server is analyzed, considering its events, use
cases, dynamic operation and the flow of events.

\subsubsection{User mock-ups}
\label{sec:user-mockups-2}
Fig.~\ref{fig:user-mockups-rs} illustrates the user mock-ups for the
\texttt{Remote Server}. It intends to mimic the user interaction with the
\texttt{Remote server} interface,
clarifying the user actions and the respective responses, as well as the
workflow, defining the \texttt{Remote Server} interface.
%
\begin{figure}[htb!]
\centering
    \includegraphics[width=1.0\columnwidth]{./img/user-mockups-rs.png}
  \caption{User mock-ups: Remote Server}%
\label{fig:user-mockups-rs}
\end{figure}

It consists of a \gls{cli} providing basic commands to authenticate an user,
perform operations over a \gls{db} and test the operation of a designated
\texttt{Local System} (only available to administrator users).

To test the operation of a \texttt{Local System}, an \texttt{Admin} can:
\begin{item-c}
\item \emph{Normal mode}: add, delete, play or stop video, audio and fragrance;
\item \emph{Interactive\& Multimedia modes} --- camera: turn on/off the camera, apply facial
  recognition, use an image filter, take a picture or create a \gls{gif};
\item \emph{Sharing mode}: share to a designated social media network a post,
  containing a message and attachment (picture or \gls{gif}). 
\end{item-c}
%
%The initial state of the \gls{mdo-l}'s~\gls{ui} is depicted in thick border
%outline, after a \texttt{User} has been detected --- \texttt{Interaction
%  mode}. On the left it is the camera feed and on the right the commands ribbon,
%containing the hints to use the system and the available options. As it can
%been, the \texttt{User} can choose an option by hovering with pointing finger
%over the desired option for a designated amount of time (e.g., 3 seconds).
%
%The workflow can be as follows:
%\begin{item-c}
%\item If the \texttt{User} selects the \texttt{Image filter} option, the
%  \texttt{Image filtering} view is shown, presenting the options to select
%  filters (which can be scrolled through palm raising/lowering), to cancel or
%  accept the image filter. If a filter is selected \texttt{filter1\_pressed}, it
%  is applied, and if accepted it will return to \texttt{Interaction mode},
%  keeping the filter on.
%\item If the \texttt{User} selects the \texttt{Take Pic} option, \texttt{Picture
%  mode} is started with a timer to allow the \texttt{User} to get ready before
%actually taking the picture. The \texttt{User} can \texttt{Cancel} --- returning
%to main menu --- or \texttt{Share} --- starting \texttt{Sharing mode}.
%\item If the \texttt{User} selects the \texttt{Create GIF} option, \texttt{GIF
%    mode (setup)} is started with a timer to allow the \texttt{User} to get
%  ready before actually creating the \gls{gif}. After the \texttt{setup\_timer}
%  is elapsed, the \texttt{GIF mode (operation)} starts, displaying a dial with
%  the \gls{gif} duration until being complete. When the \texttt{gif\_timer}
%  elapses, the \gls{gif} is created, enabling the \texttt{User} to
%  \texttt{Cancel} --- returning to main menu --- or to \texttt{Share} ---
%  starting \texttt{Sharing mode}.
%\item Lastly, in the \texttt{Sharing mode}, the \texttt{User} can
%  \texttt{Cancel} --- returning to main menu --- or select the
%  social media network. After selecting the social media, the \texttt{User} can
%  edit the post by entering its customized message and, if \texttt{Share} is
%  pressed, a message box will appear displaying the status of the post sharing
%  --- \texttt{Success} or \texttt{Error}.
%\end{item-c}
%

\subsubsection{Events}
\label{sec:events-2}
%
Table~\ref{tab:events-rs} presents the most relevant events for the
\texttt{Remote Server}, categorizing them by their source and synchrony and
linking it to the system's intended response.

\begingroup
\renewcommand{\arraystretch}{0.7} % Default value: 1
% Please add the following required packages to your document preamble:
% \usepackage{booktabs}
\begin{table}[hbt!]
\centering
\caption{Events: Remote Server}
\label{tab:events-rs}
\begin{tabular}{@{}llll@{}}
\toprule
\textbf{Event}          & \textbf{System response}                                                                                                     & \textbf{Source}                                                 & \textbf{Type} \\ \midrule
Power on                & Initialize RDBMS and go to Idle mode                                                                                         & System maintainer                                               & Asynchronous  \\ \midrule
Connection Requested    & Accept/refuse connection                                                                                                     & Remote Client                                                   & Asynchronous  \\ \midrule
Connection Accepted     & Start listening for commands                                                                                                 & Remote Client                                                   & Asynchronous  \\ \midrule
Authenticate            & \begin{tabular}[c]{@{}l@{}}Query User DB to validate user credentials.\\ If valid, login user.\end{tabular}                  & Remote Client                                                   & Asynchronous  \\ \midrule
Help                    & Send help information                                                                                                        & Remote Client                                                   & Asynchronous  \\ \midrule
Logout                  & \begin{tabular}[c]{@{}l@{}}Logout user, close connection and go to\\ Idle mode\end{tabular}                                  & Remote Client                                                   & Asynchronous  \\ \midrule
Check WiFi connection   & Periodically check WiFi connection                                                                                           & Remote Client                                                   & Synchronous   \\ \midrule
Connection timeout      & \begin{tabular}[c]{@{}l@{}}Logout user, close connection and go to \\ Idle mode\end{tabular}                                 & Remote Server                                                   & Synchronous   \\ \midrule
DB management           & Read/modify/add/delete data from DB                                                                                          & Remote Client                                                   & Asynchronous  \\ \midrule
Update stations         & Update all ready-to-run stations with ads data                                                                               & Remote Server                                                   & Synchronous   \\ \midrule
Command invalid         & Inform RC that command is invalid                                                                                            & Remote Server                                                   & Synchronous   \\ \midrule
Station notification    & Store station notification into DB                                                                                           & Local System                                                    & Asynchronous  \\ \midrule
Test Operation RC       & \begin{tabular}[c]{@{}l@{}}Parse command originated from RC and, if valid, \\ dispatch it to designated station\end{tabular} & \begin{tabular}[c]{@{}l@{}}Remote Client\\ (Admin)\end{tabular} & Asynchronous  \\ \midrule
Test Operation Callback & Provide command dispatch to original RC                                                                                      & Local System                                                    & Asynchronous  \\ \bottomrule
\end{tabular}
\end{table}

\subsubsection{Use cases}
\label{sec:use-cases-2}

\begin{figure}[htb!]
\centering
    \includegraphics[width=0.6\columnwidth]{./img/use-cases-rs.png}
  \caption{Use cases: remote server}%
\label{fig:use-cases-rs}
\end{figure}

\subsubsection{Dynamic operation}
\label{sec:dyn-oper-2}
State machine diagram

\subsubsection{Flow of events}
\label{sec:flow-events-2}
Sequence diagram

%%% Local Variables:
%%% mode: latex
%%% TeX-master: "../../../dissertation"
%%% End:
