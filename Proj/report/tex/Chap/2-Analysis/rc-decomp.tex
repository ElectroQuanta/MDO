%
\subsection{Remote Client}
\label{sec:remote-cli-decomp}
%
In this section the remote client is analyzed, considering its events, use cases, dynamic operation and the flow of events.

\subsubsection{User mock-ups}
\label{sec:user-mockups-1}
%
In Fig.~\ref{fig:user-mockups-rc} is illustrated the user mock-ups for the remote client. 
It intends to clarify how does the~\gls{ui} works for the two actors:
Brands and Company (staff).
%
\begin{figure}[htb!]
\centering
    \includegraphics[width=1.0\columnwidth]{./img/user-mockups-rc.png}
  \caption{User mockups: remote client}%
\label{fig:user-mockups-rc}
\end{figure}

The initial state of the~\gls{mdo-rc}'s~\gls{ui} is depicted in thick border outline: the 'Sign In' window. 
If the \texttt{User} makes a mistake in its username and/or password, it will be shown an error message. 
Also, the 'Sign In' window has an option to recover the password, triggering the
dispatch of an e-mail.
If the \texttt{User} still remembers its credentials, the app flows through one
out of two possibilities: if the user is an admin, goes to the admin main menu,
otherwise if the user is a brand, it will appear the brand main menu.

Firstly, the \texttt{Admin} workflow:
%
\begin{itemize}
\item The \texttt{Admin} main menu contains a drop down button with all the
  available stations. Choosing one of them, the \texttt{Admin} can turn it On/Off, see it's current mode and the current brand ad being displayed. Also, the \texttt{Admin} can log out and choose between two different paths:
%
\begin{itemize}
\item \emph{Statistics}: It is possible to see various statistics of all different brands that are currently playing on the station: the number of times that the ad was shown, the number of pictures/\gls{gif}s and shared posts, the fragrance slot and quantity (percentage) and the days remaining for the rent to end.
It is also possible to deactivate the advertisement if something wrong occurs and go back to the previous menu.
\item \emph{Users}: In this window, the \texttt{Admin} can manage all users and see their
  information, changing their type or deleting them from the database.
\item \emph{Ads to Activate}: In this window, the \texttt{Admin} can handle all the ads that the brands are intending to rent.
For that, the \texttt{Admin} needs to validate the ads': verify video's content (checking if
all videos are appropriate), if it has a filter, a fragrance and a time slot.
After that, the \texttt{Admin} can either accept or deny the ad.
If it accepts the ad, it is shown a success message and the ad is added to the station with its preferences.
Otherwise, the \texttt{Admin} indicates the denial reason, which is subsequently sent to the \texttt{Brand}'s email.
\end{itemize}
%
\end{itemize}

Secondly, the \texttt{Brand} workflow:
\begin{itemize}
\item The \texttt{Brand} main menu contains a welcome message, a notification bell to see if another ad was accepted or denied and three buttons - Rented, To Rent and Log Out.
The 'Log Out' button logs the \texttt{Brand} out of its account, the other two buttons switch to different widgets:
%
\begin{itemize}
\item \emph{Rented}: The \texttt{Brand} can see all statistics of all its rented ads on different stations that it rented.
The statistics are: status, number of times the ad was shown, the fragrance slot
and quantity (percentage), the number of pictures/\gls{gif}s taken, the number
of shared posts and the number of days remaining to end its rent.
\item \emph{To Rent}: The \texttt{Brand} can rent ads in the same station or in other stations.
For that, the \texttt{Brand} selects the target hours and then a calendar
displays the available dates. Then after choosing the days, the \texttt{Brand}
needs to upload a filter and a compressed multimedia archive with a maximum of ten videos. 
Finally, the \texttt{Brand} needs to select the fragrance and select
'Rent'. After that, a success message will be shown and the ad will enter in a
waiting queue for an \texttt{Admin} to validate.
\end{itemize}
%
\end{itemize}

It is also possible to register a new user through the 'Register' button.
This opens a window to type a username, a password, confirm the password and the e-mail.
If everything is in order, the user is created with the default user type of Brand.

Finally, at any time, it can occur the loss of internet connection, which
triggers an error message informing the automatic log out of the account.
%
\subsubsection{Events}
\label{sec:events-1}
Table~\ref{tab:events-rc} presents the most relevant events for the Local system, categorizing them by their source and
synchrony and linking it to the system’s intended response.
%
%
\begingroup
\renewcommand{\arraystretch}{0.7} % Default value: 1
%
%
\begin{table}[]
\centering
\caption{Events: remote client}
\label{tab:events-rc}
\resizebox{\textwidth}{!}{%
\begin{tabular}{@{}llll@{}}
\toprule
\textbf{Event} &
  \textbf{System response} &
  \textbf{Source} &
  \textbf{Type} \\ \midrule
Login &
  \begin{tabular}[c]{@{}l@{}}The system verifies if the user credentials\\ are correct and what type of user is and\\ asks for data from databases\end{tabular} &
  User &
  Asynchronous \\ \midrule
Verify internet connection &
  Periodically verify internet connection &
  Remote Client &
  Synchronous \\ \midrule
Statistics &
  \begin{tabular}[c]{@{}l@{}}Request to the Remote Server all the\\  information to show statistics from \\ all stations and brands\end{tabular} &
  User (Admin) &
  Asynchronous \\ \midrule
Accept/Deny ad &
  \begin{tabular}[c]{@{}l@{}}Send information to the Remote \\ Server if the ad is either accepted\\ or denied and if so, why\end{tabular} &
  User (Admin) &
  Asynchronous \\ \midrule
Power On/Off Station &
  \begin{tabular}[c]{@{}l@{}}Send command to Remote Server\\  to Power On/Off a certain station\end{tabular} &
  User (Admin) &
  Asynchronous \\ \midrule
Rented &
  \begin{tabular}[c]{@{}l@{}}Request to the Remote Server all the\\  information to show statistics from \\ all stations the brand rented\end{tabular} &
  User (Brand) &
  Asynchronous \\ \midrule
Rent &
  \begin{tabular}[c]{@{}l@{}}Send to the Remote Server all the \\ information of rent from the brand, \\ all the videos and the filter\end{tabular} &
  User (Brand) &
  Asynchronous \\ \midrule
Test Operation &
  \begin{tabular}[c]{@{}l@{}}The System dispatches the command\\ kine provided by the Remote Server\end{tabular} &
  User (Admin) &
  Asynchronous \\ \midrule
Forgot Password &
  \begin{tabular}[c]{@{}l@{}}Send e-mail to the user that has \\ forgotten his password\end{tabular} &
  User &
  Asynchronous \\ \bottomrule
\end{tabular}%
}
\end{table}
%
%
%
\subsubsection{Use cases}
\label{sec:use-cases-1}
%
Fig.~\ref{fig:use-cases-rc} depicts the use cases diagram for the \texttt{Remote Client}, describing how the system should respond under various conditions to a request from one of the stakeholders to deliver a specific
goal.
%
\begin{figure}[htb!]
\centering
    \includegraphics[width=0.6\columnwidth]{./img/use-cases-rc.png}
  \caption{Use cases: remote client}%
\label{fig:use-cases-rc}
\end{figure}

The \texttt{Admin} and the \texttt{Brand} interact with the \texttt{Remote Client} and this last interacts with the \texttt{Remote Server} to process commands, such as query databases or power on/off machines.

The \texttt{Admin} can Manage the Station, which includes Power On/Off Station, Manage Ads to Activate, Enable/Disable an Ad and test its operation.
It can also manage users, removing or modifying them. All these use cases are processed from the \texttt{Remote Client} and are requested to the \texttt{Remote Server}.

The \texttt{Brand} can see Rented Ads, Rent Ads, See notifications and register. All these cases are also processed from the \texttt{Remote Client} and are requested to the \texttt{Remote Server}.

There are some use cases that are common to the  \texttt{Admin} and to the \texttt{Brand}: Login and Logout.
%
\subsubsection{Dynamic operation}
\label{sec:dyn-oper-1}

Fig.~\ref{fig:state-mach-rc} depicts the state machine diagram for the
\texttt{Remote Client}, illustrating its dynamic behavior.
%
\begin{figure}[htb!]
\centering
    \includegraphics[width=0.9\columnwidth]{./img/state-mach-rc.png}
  \caption{State Machine Diagram: remote client}%
\label{fig:state-mach-rc}
\end{figure}
%

There are two main states:
\begin{item-c}
\item \emph{\texttt{Initialization}}: the application is initialized. The
  settings are loaded and if invalid they are restored. The WiFi communication
  is setup, signaling the communication status and if valid, an \gls{ip} address
  is returned.
\item \emph{\texttt{Execution}}: after the initialization is successful, the system goes into the \texttt{Execution} macro composite state with several concurrent activities, modeled as composite states too. However, it should be noted that there is only one actual state for the device, although at the perceivable time scale they appear to happen simultaneously. These activities are communication management (\texttt{Comm Manager}), interface management (\texttt{\gls{ui} Engine}) and application manager (\texttt{App Manager}), and are executed forever until system's power off. They are detailed next.
\end{item-c}
%
\paragraph{\emph{Communication Manager}}
Fig.~\ref{fig:state-mach-local-comm} depicts the state machine diagram for the
\texttt{Comm Manager} component. Upon successful initialization the
\texttt{Comm Manager} goes to \texttt{Idle}, listening for incoming
connections. When a remote node tries to connects, it makes a connection request
which can be accepted or denied. If the connection is accepted and the node
authenticates successfully the \texttt{Comm Manager} is ready for bidirectional
communication. When a message is received from the remote node, it is written to
\texttt{TX msg queue} and the \texttt{Supervisor} is notified. When a message
must be sent to the remote, it is read from the \texttt{TX msg queue} and sent
to the recipient. If the connection goes down, it is restarted, going into
\texttt{Idle} state again.

\begin{figure}[htb!]
\centering
    \includegraphics[width=0.5\columnwidth]{./img/state-mach-rc-comm.png}
  \caption{State Machine Diagram: remote client - Communication Manager}%
\label{fig:state-mach-rc-comm}
\end{figure}
%

\paragraph{\emph{App Manager}}
Fig.~\ref{fig:state-mach-rc-app-manag} depicts the state machine diagram for the
\texttt{App Manager} component.
Upon successful initialization the
\texttt{App Manager} goes to \texttt{Login}, waiting for some action. 
%
\begin{figure}[htb!]
\centering
    \includegraphics[width=0.9\columnwidth]{./img/state-mach-rc-app-manag.png}
  \caption{State Machine Diagram: remote client - App Manager}%
\label{fig:state-mach-rc-app-manag}
\end{figure}
%

A user can register itself by pressing the 'Register' button which leads to \texttt{Register} state: if succeeds, it returns to \texttt{Login} state. 
If the 'Login' button is pressed, the system goes to \texttt{Validation} state,
determining its type:
\begin{itemize}
\item \emph{\texttt{Admin}} --- \texttt{Admin Mode}: the \texttt{Admin} has several can
  view statistics (\texttt{Statistics}), manage all users (\texttt{Users}),
  manage all ads to actvivate (\texttt{Ads To Activate}) and test operations on
  the machines (\texttt{Test Operation}).
\item \emph{\texttt{Brand}} --- \texttt{Brand mode}: the \texttt{Brand} can see all its ads (\texttt{Rented}), see notifications and messages (\texttt{Main Menu}) and rent new ads (\texttt{To Rent}). 
\end{itemize}
%
These two states are terminated by pressing the 'Log Out' button, which
redirects to \texttt{Login} state.

If, in any state, a critical error occurs, that can cause an unexpected quit of the
\texttt{App Manager}, leading to the application abnormal shutdown.
%
%
%

\subsubsection{Flow of events}
\label{sec:flow-events-1}
The flow of events throughout the system is described using a sequence diagram, comprising the interactions between the most relevant system's entities.
It is usually pictured as the visual representation of an use case. The main
sequence diagrams are illustrated next (Fig.~\ref{fig:seq-rc-login} through Fig.~\ref{fig:seq-rc-admin-logout}).

\begin{figure}[htb!]
\centering
    \includegraphics[width=1\columnwidth]{./img/seq-rc-login.png}
  \caption{Sequence Diagram: remote client --- Login}%
\label{fig:seq-rc-login}
\end{figure}

\begin{figure}[htb!]
\centering
    \includegraphics[width=1\columnwidth]{./img/seq-rc-admin-statistics.png}
  \caption{Sequence Diagram: remote client --- admin statistics}%
\label{fig:seq-rc-admin-statistics}
\end{figure}

\begin{figure}[htb!]
\centering
    \includegraphics[width=1\columnwidth]{./img/seq-rc-admin-users.png}
  \caption{Sequence Diagram: remote client --- admin users}%
\label{fig:seq-rc-admin-users}
\end{figure}

\begin{figure}[htb!]
\centering
    \includegraphics[width=1\columnwidth]{./img/seq-rc-admin-ads-to-act.png}
  \caption{Sequence Diagram: remote client --- admin ads to activate}%
\label{fig:seq-rc-admin-ads-to-act}
\end{figure}

\begin{figure}[htb!]
\centering
    \includegraphics[width=1\columnwidth]{./img/seq-rc-admin-test-op.png}
  \caption{Sequence Diagram: remote client --- admin test operation}%
\label{fig:seq-rc-admin-test-op}
\end{figure}

\begin{figure}[htb!]
\centering
    \includegraphics[width=0.7\columnwidth]{./img/seq-rc-admin-logout.png}
  \caption{Sequence Diagram: remote client --- admin logout}%
\label{fig:seq-rc-admin-logout}
\end{figure}

As it can be seen, the \texttt{user} interacts with the \texttt{\gls{ui}}, whose events
are tracked by the \texttt{\gls{ui} Engine} triggering the appropriate callback
and dispatching data to the \texttt{Remote
  Client Back-End} for adequate processing.

There are two flow paths, pertaining to type of \texttt{User} --- \texttt{Admin}
or \texttt{Brand} --- as a result of the \texttt{User authentication}.
On both cases, it shows its main menu and it can end the sequence through the
'Logout'.

In each one of the cases there's alternative sequences to occur, depending of what the \texttt{User} decides to do.
Also, in each alternative choice, the \texttt{Remote Client} can interact with
different \texttt{Databases}, either to query or update them.
%
%%% Local Variables:
%%% mode: latex
%%% TeX-master: "../../../dissertation"
%%% End:
