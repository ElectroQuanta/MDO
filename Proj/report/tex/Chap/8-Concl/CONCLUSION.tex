	% CHAPTER - Conclusion/Future Work --------------
\chapter{Conclusion}%
\label{ch:conclusion}
In this chapter are outlined the conclusions and the prospect for future work
regarding the autonomous navigation of robots.
%
\section{Conclusions}%
\label{ch:conclusion-concls}
The autonomous navigation of robots can be described as a collision-free path
towards a target and successfully formulated and implemented using dynamic
systems. Amongst the class of the dynamic systems, the nonlinear type arises as
the most useful due to variations of qualitative behavior --- number and or/or
stability of the fixed points of the dynamic system. 

Then, the behaviors were analyzed in isolation and in integration, namely target
acquisition and obstacle avoidance, strongly supported by a previous analytical
study of the corresponding dynamic system. This analytical study, based on the
qualitative theory of dynamic systems, consisted in the determination of the
fixed points and its stability, the corresponding phase portraits, i.e.,
possible evolution of the system's state, and the bifurcation diagram, yielding
the bifurcation points as a function of the parameters of the system.
\\\\
\textbf{Target Acquisition}: An autonomous mobile robot must be able to navigate to a target position. The
target acquisition behavior can be yielded by dynamic systems for the heading
direction and path velocity (Eq.~(\ref{eq:3}) and Eq.~(\ref{eq:4})). The heading direction dynamics consists of a
sinusoidal function (Eq.~(\ref{eq:5})), as it is required the heading direction
converges to the target independently of its initial state, yielding an attractor in the heading direction and a
repeller in the opposite one for faster convergence. 

The implementation of the
dynamic systems was performed using Forward Euler's method in Matlab, and used
to command a simulation in CoppeliaSim environment. Several scenarios were
simulated for different velocities functions, and was observed that a dynamic
system for path velocity is required for adequate velocity control. Alongside
with these simulations, the system's parameters were tuned, namely the time
constants of the system. 

Lastly, a linear dynamic system for heading direction
was considered and compared to the nonlinear one. 
It was observed the nonlinear
dynamic system is better suited for the target acquisition behavior of the
robot, as it provides smoother paths --- narrower angular velocity range.
\\\\
\textbf{Obstacle avoidance}: While moving to a target, a robot must also avoid obstacles that may appear ---
obstacle avoidance. To avoid obstacles, the robot must firstly detect them, in
this case, using infrared radiation sensors. The strategy adopted consisted in
assuming that each sensor $i$ specifies a virtual obstacle in the direction
$\psi_{obs,i}$ if an obstruction is detected in that direction, modelled by a
repulsive force centered in the direction the respective sensor points out,
$f_{obs,i}$ (Eq.~(\ref{eq:24})). As each sensor is mounted in a fixed direction
$\theta_i$ in respect to the frontal direction of the robot, the calibration of
the system is unnecessary. 

The obstacle avoidance dynamics establishes a
repeller in the virtual obstacle direction with variable strength, as a function
of the distance to obstacles (Eq.~(\ref{eq:25})) --- high for small distances
and low for high distances --- and where the angular range over which the
force-let exerts its repulsive effect should be limited
(Eq.~(\ref{eq:26})). Gaussian white noise is also added to the dynamics to
enable the escape from the repeller in a finite time, in case this is the
initial condition. 

The implementation was performed in Matlab using Forward
Euler's method. Several simulations were then performed to evaluate the impact
of parameters' variations, namely the maximum strenght of repulsion,
$\beta_1$, the decay rate of the repulsion force with the distance increase,
$\beta_2$, and noise magnitude $Q$. $\beta_1$ relates to the minimum time
constant of the obstacle avoidance dynamics $\tau_{obs,i}$ (Eq.~(\ref{eq:32})),
and for the simulations performed $\beta_1 = 3.5 \Delta t$. It should be noted,
however, that the value of this parameter is dependent on the rate of fixed
points shift. As for the parameter $\beta_2$, for increasing values the decay
rate diminishes, maintaining the repulsive effect significantly for
a wider distance range --- the robot starts to rotate farther away from the
obstacles. Conversely, for decreasing values of $\beta_2$, the decay rate
increases, maintaining the repulsive effect significantly for
a narrow distance range --- the robot starts to rotate closer to the
obstacles. The addition of noise is important, but should be limited in
magnitude to avoid jitter: it should be sufficient enough to guarantee the escape from the
repellers within a time limit, in case the system is initially placed
there. Additionally, noise induces different qualitative behaviors (turn
left/right) due to stochastic effect. It was also noted that noise may not
always be required, as the system may present slight bias in sensors
readings. 

The simulations with different gap between obstacles showed a
the heading direction dynamics
exhibits different qualitative behaviors --- the stability of the fixed
points varies --- starting from gap = 50 cm. This corresponds to a bifurcation
point, representing the distance below which the robot
(with diameter 45 cm) cannot pass between the two obstacles. For gap < 50 cm, the planning dynamics has an reppeller
at the heading direction $\phi = \pi/2$, and for gap > 50 this reppeller becomes
asymptotically stable (i.e., an attractor). 

Lastly, a simulation was performed
in an environment with multiple obstacles, with the robot successfully and
adequately avoiding all of them.
\\\\
\textbf{Integration: Obstacle avoidance and Target acquisition}:
Next, the obstacle avoidance and target acquisition behaviors were integrated,
considering two approaches for the latter: nonlinear or linear dynamic
system. For both approaches the obstacle avoidance behavior must take precedence over the target acquisition
to prevent the robot from hitting an obstacle while attempting to move to the
target. This is guaranteed by imposing greater magnitude for repulsive component than for
attractive one, i.e. $\lambda_{obs} \gg \lambda_{tar}$, and consequently,
$\tau_{obs} \ll \tau_{tar}$.
After analyzing the robot's behavior for both nonlinear and linear dynamics, it
is was possible to compare them.
Mathematically, the fact that the nonlinear function is a sinusoidal function, yielding always
two fixed points --- one attractor, and in the opposite direction a
repeller --- whereas the linear function only contributes with one attractor to
the vector field. 
The repeller in the opposite direction to the target reinforces the heading direction that must be taken by the robot so that it reaches the target.
In the previous experiments, as well by Section~\ref{sec:discussion-linear-phi},
it is concluded by experimentation too, that the nonlinear system provides smoother paths at narrower angular velocity range.
Thus, it can be concluded that the nonlinear dynamic system is better for the target acquisition behavior of the robot. 
%
\\\\
\textbf{Control of driving speed}
The stability of the planning dynamics is affected by the rate of fixed points
shift, as to keep the system stable, i.e. in or near an attractor at all times, the rate of such
shifts must be limited to permit the track the attractor as it shifts. One way
of accomplishing this is by controlling the path velocity of the vehicle, as the
rate of fixed points shift is determined by the relative velocity of the robot
with respect to its environment~\cite{bicho2000dynamic}.

The shifting of fixed points for the planning dynamics can stem from robot
movement through the environment and associated sensory information 
changes or due to environmental changes (obstacles moving in the world), causing
attractors and repellers to change. 

Thus, a more adequate dynamic system for path velocity was
established, as the sum of obstacles and target contributions, where one of the
components dominates at all times. A systematic way to construct a
function that indicates if obstacles contributions are present, is to integrate
force-lets, from which a potential function of the obstacle avoidance dynamics
results. If the potential is positive a repeller is established for the planning
dynamics, and conversely, if negative an attractor arises. A convenient way 
to transform the potential levels to the strengths of the two contributions
to the velocity control is through the use of a sigmoidal threshold function,
defining the angular range the obstacles contribution is noticeable.

Then, the parameters were tuned considering the hierarchy of relaxation rates
--- which ensures that the system relaxes to the attractors and that obstacle
avoidance has precedence over the target --- and some rules of thumb for the
remaining path velocity parameters.

Finally, two scenarios were simulated --- S and Tar-Obs --- to assess the
performance of the overall dynamics. It was shown that, although the average
velocity was increased, the planning dynamics remains robust, suggesting a performance improvement in the overall dynamics.
%
\\\\
\textbf{Final remarks}:
Dynamic systems can be used to successfully implement the autonomous mobile
navigation of robots in a robust way, as the system follows closely the
asymptotically stable states, i.e. the attractors, yielding it robust to noise
and external disturbances. Specifically, nonlinear dynamic systems provide a
richness of qualitative behaviors as required for the highly dynamic conditions
imposed by the surroundings. Although more complex than the linear counterparts,
the qualitative theory of dynamic systems provides a convenient framework for a
relatively easy analysis of nonlinear dynamic systems.

The autonomous navigation of robots can be described as a collision-free path
towards a target, thus requiring both obstacle avoidance and target acquisition
behaviors for the planning dynamics. For the former, a repeller is established in the direction of the
virtual obstacles detected by the robot's sensor; for the latter, an attractor
is established in the direction of the target and a repeller in the opposite
direction for faster convergence of the desired heading direction. In the
integration of both behaviors the hierarchy of time constants must be observed,
assuring obstacle avoidance precedence over target acquisition and in a adequate
form, related to the computing time of the system.

To keep the planning dynamics stable, i.e. in or near an attractor at all times, the rate of such
shifts must be limited to permit the track the attractor as it shifts. One way
of accomplishing this is by controlling the path velocity of the vehicle, as the
rate of fixed points shift is determined by the relative velocity of the robot
with respect to its environment~\cite{bicho2000dynamic}. Thus, a more adequate
dynamic system for path velocity can be established using obstacles and target
contributions to the dynamic through a thresholded potential function and a
convenient tuning of the system.
%
\section{Prospect for Future Work}%
\label{ch:conclusion-future-work}
In the foreseeable future, the deployment of the planning and path velocity
dynamics to hardware could be performed, analyzing the navigation behavior of
the robot in real world scenarios. Additionally, the detection of targets could
be performed by the robot using different techniques such as sound detection or
computer vision.

Another interesting feature is to incorporate artificial intelligence to the
robot, enabling it to recall the most relevant target and to forget the others,
making it even more adaptable to the highly dynamic conditions of the surroundings. 
%%% Local Variables:
%%% mode: latex
%%% TeX-master: "../../dissertation"
%%% End:
