	% CHAPTER - Conclusion/Future Work --------------
\chapter{Conclusion}%
\label{ch:conclusion}
In this chapter are outlined the conclusions and the prospect for future work
regarding the marketing digital outdoor.
%
\section{Conclusions}%
\label{ch:conclusion-concls}
The Marketing Digital Outdoor can be described as a multisensory marketing device, developed to help
brands to share their products in a more efficient and attractive way through a
photo booth with gesture recognition and face based filters, video and sound
display, pictures and gif taking and also fragrance emission.

This project followed closely the waterfall methodology, which assists the
designer to quickly and systematically develop cyber physical systems.

In the analysis stage, the foreseen specifications were listed, as well as the
envisioned tests for verification and validation of the product.
The analysis elicited the main requirements and constraints about the problem at
hands, easing the conception of a design solutions pool. It was clear from early
on that the development of a distributed architecture was obviously the most
extensible and efficient way to solve the problem statement.

In the design phase, the considerations drawn in the analysis, combined with the
initial design, were used to conceptualize a viable solution for the product
materialization, comprising three main components: remote client, remote
server and local system. 

The implementation phase mapped the devised solution into actual software and
hardware modules for the several subsystems. The Remote Client was successfully
implemented, enabling a Brand or a Admin user to interact with the MDO device
with a scalable structure, consisting of a UI and MySQL database. The Local
System implemented all the major features proposed namely multisensory marketing
with video and audio
reproduction, fragrance diffusion, user interaction through a non-physical
interface (gestures), providing image filtering, picture and GIF sharing
features. Additionally, Twitter sharing was added to share posts as a way to increase
brand awareness.
Related to the technical details, a distributed architecture was successfully
implemented in a multithreaded way, enabling multiple concurrent actions to be
performed without impact to system performance from the user perspective. Device
drivers, daemons, message queues, mutexes, conditional variables were used for
this purpose.

Considering the deployment, the process was well established considering the
custom embedded Linux OS generation and initialization. However, problems in the
cross compilation of the application prevented the successfully deployment of
the application on the target hardware on a timely manner.

The tests performed over the hardware and software components demonstrated the
implemented solution performed according to the expectations, although not in
the target hardware conceived.

From a critical point of view, this project provides a solid base for developing
robust embedded systems, as the process, although laborious and complex, was
well supported, proving the concept. The deployment is lacking, but, is the
authors opinion, that this is only a question of time, as the logic behind was proven.

%
\section{Prospect for Future Work}%
\label{ch:conclusion-future-work}
In the foreseeable future, the deployment of the local system to the target
hardware must be performed and intensively tested to analyze system behavior and
performance. Twitter sharing with media upload is lacking due to Twitter API
changes and is an important feature. Additionally, the gesture detection engine
could be improved by training more accurate models.

On the Remote System side, the Remote server must be completed and deployed
externally as part of some cloud-based solution.
%%% Local Variables:
%%% mode: latex
%%% TeX-master: "../../dissertation"
%%% End:
