\chapter{Testing}
\label{cha:testing}
In this section the hardware and software tests performed are described on a
unit and integrated level.

\section{Remote Client}
\label{sec:test-rc}

In the Remote Client application, there were many test cases to implement in order to verify all the integrity of the app.
There are two main types that can be distinguished: layout tests and logic tests.

\subsection{Layout}
\label{subsec:layout-test-rc}

On the layout tests, the main tests were functional, such as try to navigate between windows, verify some critical decisions of the user, prevent some bugs on some button click, and so on.

On fig~\ref{fig:rc-quit-test} shows the behaviour of the application when the user clicks on the exit button, while on fig~\ref{fig:rc-logout-test} shows the behave to a user trying to log out. These are just some examples of test cases on the layout that turned out to behave as expected, such as the others not mentioned.

\begin{figure}[htb!]
  \centering
  \begin{subfigure}{.45\textwidth}
    \includegraphics[width=\textwidth]{img/rc-quit-test.jpg}%
  \caption{User try to quit Test Case}%
  \label{fig:rc-quit-test}
  \end{subfigure}
  % 
  \begin{subfigure}{.45\textwidth}
    \includegraphics[width=\textwidth]{img/rc-logout-test.jpg}%
  \caption{User try to logout Test Case}%
  \label{fig:rc-logout-test}
  \end{subfigure}
  % 
  \caption{\gls{mdo-l} Layout Test Cases}%
  \label{fig:rc-layout-test}
\end{figure}

\subsection{Logic}
\label{subsec:logic-test-rc}

On the logic tests, there were two different types of tests: the ones who were only with Qt based inputs and classes and error handling with queries on MySQL Server.

\subsubsection{Interface based tests}
The interface based tests are related to validate some kind of inputs before the data processing or data sent, in order to minimize errors or unexpected scenarios as much as possible. Two examples of this type of tests can be shown on the register view, where the user needs to input a valid email with the character '@' and the string ".com" (Fig.~\ref{fig:rc-email-test}) and the need to validate the password (Fig.~\ref{fig:rc-password-test}).
%
\begin{figure}[htb!]
  \centering
  \begin{subfigure}{.45\textwidth}
    \includegraphics[width=\textwidth]{img/rc-email-test.jpg}%
  \caption{Email Register Test Case}%
  \label{fig:rc-email-test}
  \end{subfigure}
  % 
  \begin{subfigure}{.45\textwidth}
    \includegraphics[width=\textwidth]{img/rc-password-test.jpg}%
  \caption{Confirm Password on Register Test Case}%
  \label{fig:rc-password-test}
  \end{subfigure}
  % 
  \caption{\gls{mdo-l} Logic (interface based) Test Cases}%
  \label{fig:rc-interface-test}
\end{figure}

\subsubsection{Error handling tests (MySQL)}
The error handling tests are related mainly with queries with MySQL that can not return as expected. One good example of that is on the login menu, because if there's no username and password matching the input, the SQL server will not return any type of data, and in this case it is necessary to handle that error and process it in the best way possible.

Fig.~\ref{fig:rc-login-test} depicts the example described above.
%
\begin{figure}[!htb]
    \includegraphics[width=.5\textwidth]{img/rc-login-test.jpg}%
  \caption{Bad Login Test Case}%
  \label{fig:rc-login-test}
\end{figure}

Also, tests related to ad a new user can be made to verify if a user is created. Fig~\ref{fig:rc-new-register-test} shows an example of that
%
\begin{figure}[htb!]
  \centering
  \begin{subfigure}{.45\textwidth}
    \includegraphics[width=\textwidth]{img/rc-new-register-test.jpg}%
  \caption{New Register Test Case}%
  \label{fig:rc-email-test}
  \end{subfigure}
  % 
  \begin{subfigure}{.45\textwidth}
    \includegraphics[width=\textwidth]{img/rc-new-register-sql-test.jpg}%
  \caption{Confirm Register on SQL Test}%
  \label{fig:rc-password-test}
  \end{subfigure}
  % 
  \caption{Register Test Case}%
  \label{fig:rc-new-register-test}
\end{figure}

\section{Remote Server}
\label{sec:test-rs}

In the Remote Server application, the tests that were needed to make were exclusively with the good creation of the database, the connection between local system and remote client and the upload and update local system.

In Fig.~\ref{fig:rs-db-test} is the proof that the database was well created with the scripts created.
%
\begin{figure}[!htb]
    \includegraphics[width=.5\textwidth]{img/rs-db-test.jpg}%
  \caption{Database Test Case}%
  \label{fig:rs-db-test}
\end{figure}

Now, on Figure~\ref{fig:rs-server-test} is a validation of the connection between local system and remote client.
%
\begin{figure}[!htb]
    \includegraphics[width=.7\textwidth]{img/rs-ls-test.jpg}%
  \caption{Communication Between systems Test Case}%
  \label{fig:rs-server-test}
\end{figure}

Lastly, on Fig.~\ref{fig:rs-upload-test} and Fig.~\ref{fig:rs-download-test} is the Test case of the update of the local system uploading the file of a new Ad.
%
\begin{figure}[htb!]
  \centering
  \begin{subfigure}{.7\textwidth}
    \includegraphics[width=\textwidth]{img/rs-upload-test.jpg}%
  \caption{Send and upload of file}%
  \label{fig:rs-upload-test}
  \end{subfigure}
  % 
  \begin{subfigure}{.6\textwidth}
    \includegraphics[width=\textwidth]{img/rs-download-test.jpg}%
  \caption{Download of file}%
  \label{fig:rs-download-test}
  \end{subfigure}
  % 
  \caption{Update and upload Test}%
  \label{fig:rs-update-upload-test}
\end{figure}
%%% Local Variables:
%%% mode: latex
%%% TeX-master: "../../../dissertation"
%%% End:

\section{Local System}
\label{sec:test-ls}

The Local System is the most critical and more susceptible subsystem to cause errors. The probability of one error occur is bigger than on the other two subsystems and that's why this system is the one that needs to perform more test cases to be in the best performance possible.

The test cases can be divided in two main parts: hardware tests and software tests.

\subsection{Hardware Tests}
\label{subsec:ls-hw-tests}
%
There are many parts of hardware that need to be tested:
\begin{item-c}
\item Ultrasonic Sensors;
\item Fragrance Diffusion (Actuator and Module);
\item Camera;
\item Speakers;
\item Screen;
\end{item-c}

\subsubsection{Ultrasonic Sensors}
\label{sec:ussensors}

The ultrasonic sensors need to be with strong connections to all the supply sources and \gls{gpio} pins, after that, it has been ran a driver program that returned if the two sensors detected the presence of an obstacle. 
This program has a specific sample time that is more than enough to detect the presence of something in front of the sensors.
In fig.~\ref{fig:sensors-test} is the cable management of the sensors, while in fig.~\ref{fig:sensors-out-test} is the test output that ran as expected.
%
\begin{figure}[htb!]
  \centering
  \begin{subfigure}{.4\textwidth}
    \includegraphics[width=\textwidth]{img/sensors-test.jpg}%
  \caption{Sensors Cable Management}%
  \label{fig:sensors-test}
  \end{subfigure}
  % 
  \begin{subfigure}{.5\textwidth}
    \includegraphics[width=\textwidth]{img/sensors-out-test.jpg}%
  \caption{Sensors Output Test}%
  \label{fig:sensors-out-test}
  \end{subfigure}
  % 
  \caption{Ultrasonic Sensors Test Cases}%
  \label{fig:uss-test}
\end{figure}

\subsubsection{Fragrance Diffusion (Actuator and Module)}
\label{sec:frag}

The fragrance diffuser module and its respective actuator need to be tested in order to respond to some signals gave by the main board.
In order for this to happen, it was tested with a driver program on the board and with all the module and it's components and the result was, as it can be seen in Fig.~\ref{fig:frag-test} what one expected.

\begin{figure}[!htb]
    \includegraphics[width=.3\textwidth]{img/frag-test.jpg}%
  \caption{Fragrance Diffuser Output Test}%
  \label{fig:frag-test}
  \end{figure}
  
  
\subsubsection{Camera}
\label{sec:camera-test}
  
For the interaction mode it is mandatory that the camera module works properly to take pictures and gifs.
For this piece of hardware the test case is simple: turn on the camera and try to take a photo.

In this case it was used an image of Raspbian and with the help of an online app, the test of the camera worked as well as expected.
The result is on Fig.~\ref{fig:camera-test}.

\begin{figure}[!htb]
    \includegraphics[width=.45\textwidth]{img/camera-teste.jpg}%
  \caption{Camera Output Test}%
  \label{fig:camera-test}
  \end{figure}
  
\subsubsection{Speakers}
\label{sec:speakers-test}

The speakers test cases are in a certain way different to test, because there's no way to show how it worked. However, the test was as simple as connect the speakers to the screen module board and play a video or an audio.

The audio played perfectly and the sound was well detected by human ears.

\subsubsection{Screen}
\label{sec:screen-test}

Testing the screen is similar to test the speakers. It's just simply connect the screen to the board and test its execution. As it can be seen in the previous figure when testing the camera (Fig.~\ref{sec:camera-test}) the screen was already being tested and it's more than proved that the screen works with no problems.


In conclusion, all hardware components and modules were tested successfully, which means that all test cases are now validated and it is possible to take the next step.
%%% Local Variables:
%%% mode: latex
%%% TeX-master: "../../../dissertation"
%%% End:


\section{Software}
\label{sec:software}
In this section the software tests conducted over the \texttt{Remote Client},
\texttt{Remote Server}, \texttt{Database} and \texttt{Local System} are presented.

\subsection{Local System}
\label{sec:local-system}
The \texttt{Local system} tests were performed on a host computer, prior to its
deployment to the \texttt{Raspberry Pi}. These tests are described next.

\subsubsection{Computer vision}
\label{sec:computer-vision-1}
In this section are described the frame acquisition,
face detection, and gesture recognition tests. Fig.~\ref{fig:cv-tests}
illustrates the combination of these tests. It can be seen that the camera
frames are acquired and processed to detect multiple faces and apply filters overlay, and
also to detect gestures that can be used to trigger UI events.
Additionally, it can be seen a picture that was taken, stored and displayed,
which is ready for sharing on social media.
\begin{figure}[htb!]
\centering
    \includegraphics[width=0.6\textwidth]{./img/UI-test-filters.png}
  \caption{Computer visions tests}%
\label{fig:cv-tests}
\end{figure}

\subsubsection{Normal mode}
\label{sec:normal-mode}
Fig.~\ref{fig:normal-mode-test} illustrates the normal mode testing. It
demonstrated that a video can be reproduced on a loop while the fragrance
diffuser was also enabled and disabled according to its on and off times. The
normal mode only exited if there was no current ad enabled or if a user was
detected.
%
\begin{figure}[htb!]
\centering
    \includegraphics[width=0.6\textwidth]{./img/ui-test-normal-mode.png}
  \caption{Normal mode: testing}%
\label{fig:normal-mode-test}
\end{figure}

\subsubsection{Interaction mode}
\label{sec:interaction-mode}
The interaction mode tests demonstrated that it was possible to take pictures
and create GIFs on demand, as illustrated in Fig.~\ref{fig:cv-tests}.

\subsubsection{Twitter sharing}
\label{sec:twitter-sharing}
Fig.~\ref{fig:twitter-test} illustrates the Twitter sharing mode. It
demonstrated that a post can be successfully shared from the Local System into
Twitter. Nonetheless, at the current time, it is still not possible to upload
media to Twitter, as the API recently changed, requiring extra time to implement.
%
\begin{figure}[htb!]
  \centering
  % 
  \begin{subfigure}[t]{.4\textwidth}
  \includegraphics[width=\textwidth]{img/ui-test-sharing-mode.png}%
  %\caption{main}%
  %\label{fig:state-mach-local-superv-main}
\end{subfigure}
%
\hspace{.05\textwidth}
%
  \begin{subfigure}[t]{.42\textwidth}
  \includegraphics[width=\textwidth]{img/ui-test-sharing-mode2.png}%
  %\caption{Request Handler}%
  %\label{fig:state-mach-local-superv-req}
\end{subfigure}
  % 
  \caption{Twitter sharing: testing}%
  \label{fig:twitter-test}
\end{figure}
%

\subsubsection{Image filtering mode}
\label{sec:image-filtering-mode}
From Fig.~\ref{fig:cv-tests} and Fig.~\ref{fig:twitter-test}, it is possible to
verify that several filters were successfully implemented. Moreover, it is
possible to observe from the last figure that the image filter can be persistent
if desired.

\subsubsection{Download Ad}
\label{sec:download-ad}
Fig.~\ref{fig:download-test} illustrates the download of an Ad in the
background, while the application was currently busy in the interaction mode.
%
\begin{figure}[htb!]
\centering
    \includegraphics[width=0.6\textwidth]{./img/ui-test-download-ad.png}
  \caption{Download Ad: testing}%
\label{fig:download-test}
\end{figure}


\subsubsection{Connection to Remote System}
\label{sec:conn-remote-syst}
The connection to the remote system was simulated and tested deploying a server
listening in the \texttt{localhost} and observing the connection and data exchange between the two
nodes. This test was successful validating the client-server architecture logic
and implementation.

\subsubsection{User detection}
\label{sec:user-detection}
The user detection was tested by deploying the user detection daemon to
stimulate the application. It was verified that the daemon successfully detects
the ultrasonic sensors triggering when a person passes and that this event is
conveyed to the local system via message queue where it is successfully
detected, signaling a user was detected.

%%% Local Variables:
%%% mode: latex
%%% TeX-master: "../../../dissertation"
%%% End:
