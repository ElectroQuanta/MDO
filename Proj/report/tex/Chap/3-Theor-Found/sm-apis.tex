\section{Social media sharing APIs}
\label{sec:social-media-sharing}
Social media platforms are great contact points to target customers and to
increase brand awareness, which is highly desirable for digital marketing. These
platforms provide a set of functionalities to external agents interact with them
in a programmatic way, enabling automation of tasks like content sharing,
scheduling, search, etc.
These functionalities are exposed by \glspl{api}, providing a custom and
well-defined interface that can be explored by developers to build custom
applications that leverages on them.

There are several social media platforms, but here one focuses on the most
popular ones, namely~\cite{rakutenTop10SMApis, ayrshareTop10SMApis}:
\begin{enum-c}
\item \emph{Facebook}:
  Facebook is a social networking platform that allows users to communicate
  using messages, photos, comments, videos, news, and other interactive content.
  \begin{itemize}
  \item \emph{API features}:
    Facebook provides various \glspl{api} and \glspl{sdk} that allow developers to access its
    data and extend the capabilities of their applications. The Facebook Graph
    API is an \gls{http}--based API that provides the main way of accessing the
    platform's data. With the API, you can query data, post images, access
    pages, create new stories, and carry out other tasks. Furthermore, the
    Facebook Marketing API allows you to create applications for automatically
    marketing your products and services on the platform.
  \item \emph{Popularity}:
    At the end of 2018, Facebook was boasting of more than 2.2
    billion monthly active users, making it the most popular social media
    platform in the world.
  \item \emph{Price}:
    The Facebook APIs --- Graph API --- are provided for free.
  \item \emph{Ease of use}:
    Apart from its detailed documentation, Facebook has an active
developer community with members who are always willing to assist each other
make the most of them. The API is large, documentation decent and leans towards
PHP, and requires the developer to create a video or screencast to be approved.
\item \emph{Registration process}: the developer needs to register its app, and if for
  commercial purposes, register its business. This includes verifying his/her legal entity and address.
  \end{itemize}
%
\item \emph{Instagram}:
Instagram is a Facebook-owned social networking platform that lets users share
photos and videos.
%
\begin{itemize}
\item \emph{API features}:
  Facebook offers many APIs to allow developers to create tools that enhance
  users' experience on the Instagram platform. With the APIs, you can enable
  users to share their favorite stories and daily highlights from your
  application to Instagram. Furthermore, there is the Instagram Graph API that
  allows developers to access the data of businesses operating Instagram
  accounts. With the Graph API, you can conveniently manage and publish media
  objects, discover other businesses, track mentions, analyze valuable metrics,
  moderate comments, and search hashtags.
\item \emph{Popularity}:
  At the end of 2018, Instagram had more than 1 billion monthly active users.
\item \emph{Price}:
  The APIs are offered for free.
\item \emph{Ease of use}:
  The Instagram APIs are easy to use. Facebook has done good
  work in providing detailed documentation to assist developers in easily
  implementing the APIs into their applications.
\item \emph{Registration process}: subject to the same terms of Facebook
  registration process. Additionally, the API allows the developer to pull data, but it can
  not post via the API unless he's/she's a `Partner'. The partner
 program started in 2018 and isn't open to new partners unless Instagram asks
 one to join.  
\end{itemize}
%
\item \emph{Twitter}:
Twitter is a popular social media service that allows users to find the latest
world events and interact with other users using various types of messaging
content (called tweets). Twitter can be accessed via its website interface,
applications installed on mobile devices, or a short message service (SMS).
\begin{itemize}
\item \emph{API features}:
  Twitter provides various API endpoints for completing various tasks. For example, you can use the Search API to retrieve historical tweets, the Account Activity API to access account activities, the Direct Message API to send direct messages, Ads API to create advertisement campaigns, and Embed API to insert tweets on your web application.
\item \emph{Popularity}:
Twitter is a very popular social media networking service that can assist in enhancing the engagement of your application. At the end of 2018, it had more than 335 million monthly active users.
\item \emph{Price}:
Twitter provides its APIs for free. However, if you want a high level of access and reliability, you’ll need to contact them for paid API versions.
\item \emph{Ease of use}:
Ease of use: The Twitter APIs are very easy to use. Twitter provides
comprehensive documentation to assist you in flawlessly integrating any of its
APIs into your specific use case. Some API wrappers are also available in
several programming languages. The new Twitter developer site no longer allows
localhost in their approved callback URLs, so you’ll need to use \texttt{ngrok} or an equivalent tunnel to test locally.
\item \emph{Registration process}: it requires the application to be registered as a
  new project (using Twitter's Developer interface) and the developer can get a
  permanent access token. Much easier than for the other two platforms.
\end{itemize}
\end{enum-c}

The social media \glspl{api} are specific to each platform. Thus, developing a
custom interface for each of these platforms is a very time consuming
task. Instead, one will focus only on one social media platform, and its
associated \gls{api}. The chosen platform is \textbf{Twitter} as it's the less
extent and more easy to implement and its registration process is the least cumbersome.

\subsection{Twitter API}
\label{sec:twitter-api}




%%% Local Variables:
%%% mode: latex
%%% TeX-master: "../../../dissertation"
%%% End:
