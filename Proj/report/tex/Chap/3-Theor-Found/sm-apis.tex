\section{Social media sharing APIs}
\label{sec:social-media-sharing}
Social media platforms are great contact points to target customers and to
increase brand awareness, which is highly desirable for digital marketing. These
platforms provide a set of functionalities to external agents interact with them
in a programmatic way, enabling automation of tasks like content sharing,
scheduling, search, etc.
These functionalities are exposed by \glspl{api}, providing a custom and
well-defined interface that can be explored by developers to build custom
applications that leverages on them.

There are several social media platforms, but here one focuses on the most
popular ones, namely~\cite{rakutenTop10SMApis, ayrshareTop10SMApis}:
\begin{enum-c}
\item \emph{Facebook}:
  Facebook is a social networking platform that allows users to communicate
  using messages, photos, comments, videos, news, and other interactive content.
  \begin{itemize}
  \item \emph{API features}:
    Facebook provides various \glspl{api} and \glspl{sdk} that allow developers to access its
    data and extend the capabilities of their applications. The Facebook Graph
    API is an \gls{http}--based API that provides the main way of accessing the
    platform's data. With the API, one can query data, post images, access
    pages, create new stories, and carry out other tasks. Furthermore, the
    Facebook Marketing API allows one to create applications for automatically
    marketing one's products and services on the platform.
  \item \emph{Popularity}:
    At the end of 2018, Facebook was boasting of more than 2.2
    billion monthly active users, making it the most popular social media
    platform in the world.
  \item \emph{Price}:
    The Facebook APIs --- Graph API --- are provided for free.
  \item \emph{Ease of use}:
    Apart from its detailed documentation, Facebook has an active
developer community with members who are always willing to assist each other
make the most of them. The API is large, documentation decent and leans towards
PHP, and requires the developer to create a video or screencast to be approved.
\item \emph{Registration process}: the developer needs to register its app, and if for
  commercial purposes, register its business. This includes verifying his/her legal entity and address.
  \end{itemize}
%
\item \emph{Instagram}:
Instagram is a Facebook-owned social networking platform that lets users share
photos and videos.
%
\begin{itemize}
\item \emph{API features}:
  Facebook offers many APIs to allow developers to create tools that enhance
  users' experience on the Instagram platform. With the APIs, one can enable
  users to share their favorite stories and daily highlights from one's
  application to Instagram. Furthermore, there is the Instagram Graph API that
  allows developers to access the data of businesses operating Instagram
  accounts. With the Graph API, one can conveniently manage and publish media
  objects, discover other businesses, track mentions, analyze valuable metrics,
  moderate comments, and search hashtags.
\item \emph{Popularity}:
  At the end of 2018, Instagram had more than 1 billion monthly active users.
\item \emph{Price}:
  The APIs are offered for free.
\item \emph{Ease of use}:
  The Instagram APIs are easy to use. Facebook has done good
  work in providing detailed documentation to assist developers in easily
  implementing the APIs into their applications.
\item \emph{Registration process}: subject to the same terms of Facebook
  registration process. Additionally, the API allows the developer to pull data, but it can
  not post via the API unless he's/she's a `Partner'. The partner
 program started in 2018 and isn't open to new partners unless Instagram asks
 one to join.  
\end{itemize}
%
\item \emph{Twitter}:
Twitter is a popular social media service that allows users to find the latest
world events and interact with other users using various types of messaging
content (called tweets). Twitter can be accessed via its website interface,
applications installed on mobile devices, or a short message service (SMS).
\begin{itemize}
\item \emph{API features}:
  Twitter provides various API endpoints for completing various tasks. For example, one can use the Search API to retrieve historical tweets, the Account Activity API to access account activities, the Direct Message API to send direct messages, Ads API to create advertisement campaigns, and Embed API to insert tweets on one's web application.
\item \emph{Popularity}:
Twitter is a very popular social media networking service that can assist in enhancing the engagement of one's application. At the end of 2018, it had more than 335 million monthly active users.
\item \emph{Price}:
Twitter provides its APIs for free. However, if one want a high level of access and reliability, one’ll need to contact them for paid API versions.
\item \emph{Ease of use}:
Ease of use: The Twitter APIs are very easy to use. Twitter provides
comprehensive documentation to assist one in flawlessly integrating any of its
APIs into one's specific use case. Some API wrappers are also available in
several programming languages. The new Twitter developer site no longer allows
localhost in their approved callback URLs, so one’ll need to use \texttt{ngrok} or an equivalent tunnel to test locally.
\item \emph{Registration process}: it requires the application to be registered as a
  new project (using Twitter's Developer interface) and the developer can get a
  permanent access token. Much easier than for the other two platforms.
\end{itemize}
\end{enum-c}

The social media \glspl{api} are specific to each platform. Thus, developing a
custom interface for each of these platforms is a very time consuming
task. Instead, one will focus only on one social media platform, and its
associated \gls{api}. The chosen platform is \textbf{Twitter} as it's the less
extent and more easy to implement and its registration process is the least cumbersome.

\subsection{Twitter API}
\label{sec:twitter-api}
The Twitter API can be used to programmatically retrieve and analyze Twitter
data, as well as build for the conversation on Twitter. Over the years, the
Twitter API has grown by adding more levels of access for developers and
academic researchers to be able to scale their access to enhance and research
the public conversation. Recently, the Twitter API v2 has been released,
including a modern foundation, new and advanced features, and quick on-boarding
to Essential access~\cite{twitterAbout}.

To get access to the Twitter \gls{api}, the developer needs to~\cite{twitterAPIGettingStarted}:
\begin{enum-c}
\item
  \emph{Sign up for a developer account}: After signing-up, one will create a
  Project and an associated developer App during the on-boarding process, which
  will provide a set of credentials that will be used to authenticate all
  requests to the API.
\item
  \emph{Save the Application's keys and tokens and keep them secure}:
After completing step 1, one will be able to find or generate the following
credentials within one's developer App:
\begin{itemize}
\item 
    \emph{API Key and Secret}: Essentially the username and password for one's App. One
    will use these to authenticate requests that require OAuth 1.0a User
    Context, or to generate other tokens such as user Access Tokens or an
    app-only Bearer Token.
  \item 
    \emph{A set of user Access Tokens}: In general, Access Tokens represent the user
    that one are making the request on behalf of. The ones that one can generate
    via the developer portal represent the user that owns the App. One will use
    these to authenticate requests that require OAuth 1.0a User Context. If one
    would like to make requests on behalf of another user, one will need to use
    the 3-legged OAuth flow for them to be authorized. 
  \item 
    \emph{Bearer Token}: One will use this token when making a request to an endpoint that requires OAuth 2.0 Bearer Token.
\end{itemize}

\item
  \emph{Make the first request}: after completing the first two steps, one can
  use the \gls{api} to interact with the Twitter platform.
\item
  \emph{Apply for additional access}:
  With Essential access, one is only able to make requests to the Twitter API v2
  endpoints, and not the v1.1 or enterprise endpoints.  One is limited to 500K Tweets/month, and unable to take advantage of certain developer portal functionality such as teams and access to additional App environments. 

If one wants to access the standard v1.1, premium v1.1, or enterprise endpoints,
or if one wants to take advantage of an increased Tweet cap and developer portal
functionality, one needs to apply for Elevated or Academic Research access. 
\end{enum-c}

It is important to note that the Twitter \gls{api} is a \gls{rest}
(a.k.a. \gls{rest}ful) \gls{api}, which means that when a client request is made
through it, it transfers a representation of the state of the resource to the
requester or endpoint. This information, or representation, is delivered in one
of several formats via \gls{http}: \gls{json}, \gls{html}, XLT (Excel
Templates), Python, PHP, or plain text. The Twitter API uses \gls{json} --- the most
generally popular file format to use, because, despite its name, it’s language-agnostic, as well
as readable by both humans and machines~\cite{whatsRestApi}.

\gls{rest} is a set of architectural constraints, not a protocol or a
standard. API developers can implement \gls{rest} in a variety of ways, making
REST APIs faster and more lightweight, with increased scalability --- perfect
for \gls{iot} applications and mobile app development~\cite{whatsRestApi}.

Another important note about RESTful APIs: headers and parameters are also important in the HTTP methods of a RESTful API HTTP request, as they contain important identifier information as to the request's metadata, authorization, uniform resource identifier (URI), caching, cookies, and more. There are request headers and response headers, each with their own HTTP connection information and status codes~\cite{whatsRestApi}.

For an API to be considered RESTful, it has to conform to the following
criteria~\cite{whatsRestApi}:
\begin{item-c}
\item A client-server architecture made up of clients, servers, and resources,
  with requests managed through HTTP.
\item Stateless client-server communication, meaning no client information is
  stored between get requests and each request is separate and unconnected.
\item Cacheable data that streamlines client-server interactions.
\item A uniform interface between components so that information is transferred
  in a standard form. This requires that:
  \begin{itemize}
    \item resources requested are identifiable and separate from the
      representations sent to the client.
    \item resources can be manipulated by the client via the representation they
      receive because the representation contains enough information to do so.
    \item self-descriptive messages returned to the client have enough
      information to describe how the client should process it.
    \item hypertext/hypermedia is available, meaning that after accessing a
      resource the client should be able to use hyperlinks to find all other
      currently available actions they can take.
  \end{itemize}
\item A layered system that organizes each type of server (those responsible for
  security, load-balancing, etc.) involved the retrieval of requested
  information into hierarchies, invisible to the client.
\item Code-on-demand (optional): the ability to send executable code from the
  server to the client when requested, extending client functionality.
\end{item-c}

To interact with the \texttt{Twitter} platform, one can make a request and
retrieve its response by following these steps~\cite{twitterAPIMakeFirstRequest}:
\begin{enum-c}
\item \emph{Select the end-point}: several actions can be performed on the
  Twitter website on mobile application, varying its interface.
\item \emph{Select a tool to make the request}: one can use command line tools,
  driver programs or libraries in several programming languages, or tools like
  Postman and Imsonia --- visual tools to make requests to \gls{rest} endpoints.
%
  For example, here is a \texttt{cURL} example for the user lookup endpoint.
  To use this request the \texttt{BEARER\_TOKEN} and \texttt{USERNAME} with
  Bearer Token and Twitter handle of the developer, and execute it in the
  command line.
  \begin{quote}
      \onehalfspacing
\begin{verbatim}
curl "https://api.twitter.com/2/users/by/username/$USERNAME" -H 
"Authorization: Bearer $BEARER_TOKEN"
\end{verbatim}
  \end{quote}
    \vspace{-2mm}
  %
\item \emph{Review the response}:
  Once a successful request has been made, a payload will be received with
  metadata related to the request.
  \begin{itemize}
  \item 
  If you used an endpoint that utilizes a GET HTTP method, you will receive
  metadata related to the resource (Tweet, user, List, Space, etc) that you made
  the request to in JSON format. Review the different fields that returned and
  see if you can map the information that you requested to the content on
  Twitter.
\item 
If you used an endpoint that utilizes a POST, PUT, or DELETE HTTP method, you performed an action on Twitter. Go to Twitter.com or the mobile app and see if you can track down that action. 
  \end{itemize}
%
\item \emph{Adjust the request using parameters}:
  Each endpoint has a different set of parameters that can be used to alter the
  request. For example, additional metadata fields can be requested when using
  GET endpoints with the fields and expansions parameters.
\end{enum-c}

\subsubsection{Manage Tweets example}
\label{sec:manage-tweets-exampl}
Creating and deleting Tweets using the Twitter API is essential for engaging
with the public conversation. There are two available methods to manage Tweets~\cite{twitterManageTweetIntro}:
\begin{enum-c}
\item \emph{\texttt{POST}}: it allows one to post polls, quote Tweets, Tweet
  with reply settings, Tweet with geo, Tweet with media and tag users, and Tweet to Super Followers, in addition to other
  features. It can be further customized using parameters.
There is a user rate limit of 200 requests per 15 minutes for the POST
method.
\item \emph{\texttt{DELETE}}: allows to delete a specific Tweet.
It has a rate limit of 50 requests per 15 minutes.
\end{enum-c}

Since one is making requests on behalf of a user with all manage Tweets
endpoints, one must authenticate with OAuth 1.0a User Context and use the Access
Tokens associated with a user that has authorized one's App. The Access Tokens
can be generated using the 3-legged OAuth flow.

As an example let's focus on the \textbf{post tweet} feature. For this purpose, one needs
to~\cite{twitterManageTweetQuickStart}:
\begin{enum-c}
\item \emph{Select a tool or library to make the request}:
  Postman, Imsonia,
  driver programs or libraries in several programming languages. In this case,
  one will consider \texttt{cURL} and the command line.
\item \emph{Authenticate the request}:
  To make a successful request to this endpoint, one will need to use OAuth 1.0a
  User Context. To do this, the following keys and tokens must be added to the
  shell environment by exporting the following variables:
  \begin{itemize}
    \item \texttt{consumer\_key} with your API Key
    \item \texttt{consumer\_secret} with your API Key Secret
    \item \texttt{access\_token} with your Access Token
    \item \texttt{token\_secret} with your Access Token Secret
    \end{itemize}
  \item \emph{Configure the request with parameters}: one can inspect the
    \texttt{POST /2/tweets} API call to understand its usage and
    configuration~\cite{twitterAPIRefPostTweet}. Some relevant parameters are:
    \texttt{text} --- a string containing the text of the Tweet; \texttt{media}
    --- a \gls{json} object containing the media information being attached to
    the tweet. Here, one will use just the first.
  \item \emph{Make the request}: using \texttt{cURL} one has:
    \begin{quote}
      \onehalfspacing
        \begin{verbatim}
        curl -X POST https://api.twitter.com/2/tweets -H 
        "Authorization: OAuth \$OAUTH_SIGNATURE" -H 
        "Content-type: application/json" -d '{"text": "Hello World!"}'
        \end{verbatim}
    \end{quote}
    \vspace{-7mm}
%
  \item \emph{Analyze the response}: an example response can be the following
    \gls{json} object,
    containing the \texttt{id} and the \texttt{text} of the newly created tweet.
    \begin{quote}
      \onehalfspacing
        \begin{verbatim}
        {
            "data": {
                "id": "1445880548472328192",
                "text": "Hello world!"
            }
        }
        \end{verbatim}
    \end{quote}
    \vspace{-7mm}
  \end{enum-c}

  

%%% Local Variables:
%%% mode: latex
%%% TeX-master: "../../../dissertation"
%%% End:

\subsubsection{C++ libraries and APIs}
\label{sec:c++-libraries-apis}
The Twitter \gls{api} is a \gls{rest}ful \gls{api}, which, in practical terms,
means that \gls{http} headers can be used to send (\texttt{POST}) and retrieve
data (\texttt{GET}) from the Twitter platform.

Thus, one only requires an wrapper around these \gls{http} `methods' to
interface Twitter. One such tool, as aforementioned, it's \texttt{cURL} ---
client \gls{url}, a
command-line utility for transferring data with \glspl{url}~\cite{curl}. It is a free,
open-source tool developed in C++, first released in 1997. cURL offers
`\texttt{libcurl}', a library, and `\texttt{curl}', a command-line tool, and is
often used to retrieve data from Twitter~\cite{twitterCreatingLibsC}.

Several C++ \glspl{api} are available for Twitter, although not officially
recommended by Twitter:
\begin{enum-c}
\item \texttt{kQOAuth} by Johan Paul --- a Qt based OAuth Library;
\item \texttt{libOAuth} by Robin Gareus --- a collection of POSIX-C functions implementing OAuth;
\item \texttt{QTweetLib} by Toni Jovanoski --- a Qt based Twitter API library;
\item \texttt{Twitcurl} by Mahesh --- a Twitter API library;
\end{enum-c}

There are two \texttt{Qt} based libraries, thus making it
highly-dependent on the \texttt{Qt} platform, a collection of POSIX-C functions
implementing only the authentication mechanism, and a pure C++ \gls{api} library
for Twitter. Hence, from the above two options arise for building the
\texttt{Local System}'s interface for Twitter:
\begin{enum-c}
\item \emph{\texttt{libcurl}}: a free and easy-to-use client-side URL transfer library
  with a C \gls{api}~\cite{libcurl}. One could then write a C++ wrapper around it to follow
  \gls{oop} paradigm or use the one of the recommended wrappers ---
  \texttt{curlpp}, \texttt{curlcpp}, or \texttt{C++ Requests} --- and then write
  a wrapper to handle Twitter requests~\cite{libcurlBindings};
\item \emph{\texttt{Twitcurl}}~\cite{twitcurlGithub}: use a pure C++ \gls{api} to interface \texttt{Twitter},
  providing a straightforward mechanism to handle requests. It uses
  \texttt{cURL} for handling HTTP requests and responses and it works well on
  any \gls{os} that supports \texttt{cURL}. It supports:
  \begin{itemize}
  \item v1.1 Twitter REST APIs: timeline, status, user, direct message,
    friendship, social graph, account, favorite, block, saved search and trend
    methods;
  \item \texttt{OAuth}: authentication methods for Twitter
  \end{itemize}
  \texttt{twitcurl} returns JSON responses from \texttt{twitter.com} as it
  is. Thus, a C++ JSON parser is required to parse the responses.
\end{enum-c}

\emph{Thus, for practicality reasons, \texttt{twitcurl} is the preferred option}.