%
\section{RDBMS}
\label{sec:rdbms}
An \acrfull{rdbms} is a software program used to create, maintain and manage
relational databases.
A database is a collection of well-organized related data. Examples of databases
include employee records, library management system, bus, railway, and airline
reservation systems~\cite{rdbmsDbmsDiff}.

\gls{rdbms} is a subset of \gls{dbms} with relationship between tables
and rows. It follows the relational model, introduced by E.F. Codd in 1970~\cite{ramakrishnan2003database},
instead of navigational model, where in the data is stored in multiple tables.
The tables are related to each other
using primary and foreign keys. It is the most used database model widely used
by enterprises and developers for storing complex and huge amounts of
data~\cite{rdbmsDbmsDiff}. Some examples of \gls{rdbms} are Oracle Database,
MySQL, IBM DB2, SQLite, PostgreSQL, and MariaDB.

Arguably, the most widely used form
of concurrent programming is the concurrent execution of database programs
(called transactions). Users write programs as if they are to be run by
themselves, and the responsibility for running them concurrently is given to the
\gls{dbms}~\cite{ramakrishnan2003database}.

From the users' application standpoint, a \gls{rdbms} is a management system
for databases, but is useless unless it provides an efficient and easy method to
pose questions involving the data stored in the databases. These questions are
called queries~\cite{ramakrishnan2003database}. A \gls{dbms} provides a
specialized language --- query language --- in which queries can be performed.
The \gls{sql} for relational databases, is now the standard.

In this section an overview is presented about \gls{rdbms} foundations:
relational model, levels of abstraction in a \gls{dbms}, transaction management,
and the structure of a \gls{dbms}. Additionally, a brief overview over \gls{sql}
and a C++ interface is presented.

\subsection{Relational model}
\label{sec:relational-model}

\subsection{Levels of abstraction in a DBMS}
\label{sec:levels-abstr-dbms}

\subsection{Transaction management}
\label{sec:trans-manag}

\subsection{Structure of a RDBMS}
\label{sec:structure-rdbms}

\subsection{SQL}
\label{sec:sql}

\subsection{SQL C++ interface}
\label{sec:sql-c++-interface}





%% Local Variables:
%% mode: latex
%% TeX-master: "../../../dissertation"
%% End:
