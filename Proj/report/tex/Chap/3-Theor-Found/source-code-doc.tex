%
\section{Source code documentation}
\label{sec:source-code-docum}
Source code documentation is critical for software maintenance, specially on
large developments teams. As software applications grow in size and evolve
several modifications from base source code arise, possible with several
branches to implement different features or correct bugs. A good practice that
helps to reduce the complexity burden when interfacing and maintaining such
systems is to document the source code, making it readable and easily understood
by other people.

Source code documentation aims to describe how the code works instead of what it
does, and it is useful for the following
reasons~\cite{sourceCodeDocBestPractices}:
\begin{item-c}
\item \emph{Knowledge transfer}: not all code is equally obvious. There might be
  some complex algorithms or custom workarounds that are not clear enough for
  other developers.
\item \emph{Troubleshooting}: if there are any problems with the product after
  it's released, having proper documentation can speed up the resolution
  time. Finding out product details and architecture specifics is a
  time-consuming task, which results in the additional costs.
\item \emph{Integration}: product documentation
 describes dependencies between system modules and third-party tools. Thus, it
 may be needed for integration purposes.
\item \emph{Code style enforcement}: the source code documentation tools
  requires specific comment syntax, which forces the developer to follow a
  strict code style. Having a code style standard is specially useful on large
  project teams.
\end{item-c}

There are some key ideas to write good documentation~\cite{sourceCodeDocBestPractices}:
\begin{item-c}
\item  
\emph{Simple and concise}. Follow the DRY (Don't Repeat Yourself)
principle. Use comments to explain something that requires detailed information.
\item
  \emph{Up to date at all times}: the code should be documented when it's being
  written or modified.
\item
  \emph{Document any changes to the code}. Documenting new features or add-ons is pretty obvious. However,
 you should also document deprecated features, capturing any change in the
 product.
\item
 \emph{Simple language and proper formatting}: Code documents are typically written in English so that any
 developer could read the comments, regardless of their native language. The best practices for
 documentation writing require using the Imperative mood, Present tenses, preferably active voice, and
 second person.
\end{item-c}

There are several automatic source code documentation tools, typically
language-specific, namely~\cite{sourceCodeDocBestPractices}:
\begin{item-c}
\item \emph{\texttt{Doxygen}}: C, C++, C\#, Java, Objective-C, PHP, Python
\item \emph{\texttt{GhostDoc}}: C\#, Visual Basic, C++, JavaScript
\item \emph{\texttt{Javadoc}}: Java only
\item \emph{\texttt{Docurium} or \emph{YARD}}: Ruby
\item \emph{\texttt{jsdoc}}: Javascript
\item \emph{\texttt{Sphinx}}: Python, C/C++, Ruby, etc.
\end{item-c}

\subsection{Doxygen}
\label{sec:doxygen}



%%% Local Variables:
%%% mode: latex
%%% TeX-master: "../../../dissertation"
%%% End:
