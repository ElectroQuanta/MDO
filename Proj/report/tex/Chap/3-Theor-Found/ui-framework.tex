%
\section{UI framework}
\label{sec:ui-framework}

For the development of the \texttt{Remote Client}, it is necessary to use a \gls{ui} Framework, in order to develop a \gls{ui} to the users, making it more user friendly and interactive.
There are several frameworks that can be used, such as:

\begin{itemize}
\item Qt;
\item Sciter;
\item Noesis GUI;
\item wxWidgets;
\item GTK+;
\item and so on.
\end{itemize}

For this project, \texttt{Qt} is the chosen one, for several reasons.

First of all, it is possible to "develop graphical user interfaces and cross-platform applications, both desktop and embedded".~\cite{qt-bib} 
The framework can function on different types of software and hardware.
Also, it has an easy-to-read code and it is possible to make an attractive \gls{ui}.

Secondly, this framework is \texttt{cost-friendly}, not only because it has a free license, but also because its software development takes less time to develop due to its highly productive features.

Lastly, it is implemented in C++, which means that it is possible to use many libraries.
The wide choice of modules allow the project to have rich functionality and as a result, the software will have a \gls{gui} similar to a native one.
%%% Local Variables:
%%% mode: latex
%%% TeX-master: "../../../dissertation"
%%% End:
