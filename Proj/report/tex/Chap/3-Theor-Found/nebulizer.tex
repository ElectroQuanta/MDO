%
\section{Scenting technologies}
\label{sec:scenting-techn}
A scenting technology transforms an aromatic liquid into a gaseous fluid that
can be conveyed through the air and be captured by human olfactory sense,
usually for therapeutic or marketing purposes. As aforementioned in
Section~\ref{sec:context-motivation}, olfactory sense is the fastest way to the brain, thus, providing an exceptional
opportunity for marketing~\cite{news-harvard} --- ``75\% of the emotions we generate on a daily basis are affected by smell. Next
to sight, it is the most important sense we have''~\cite{lindstrom2006brand}.

In this section a brief overview of the scenting technologies is provided, with
special focus on ultrasonic diffusion as simple to control and cost-effective
solution.

\subsection{Overview}
\label{sec:overview}
There are several scenting technologies, mainly divided into~\cite{wen2019development}:
\begin{item-c}
\item \emph{Atomization}: it dispense odorants by transforming them into a
  gaseous fluid without requiring to heat. Its advantages are
  the dispensing process is fast and the dispensing quantity is controllable.
\item \emph{Thermalization}: it dispense odorants by vaporizing odor sources in
  the liquid state of the solid state using \gls{pwm} heaters. It requires a
  temperature controller to avoid scorching odor sources.
\item \emph{Evaporation}: it dispense odorants by conveying the liquid through a
  porous material into the outer surface (capillary action) where it evaporates
  naturally. It is a passive method, thus, not controllable.
\end{item-c}

The thermalization process requires heat which can modify fragrances, besides
requiring more power. Evaporation is a passive method, hence, not
controllable. Thus, one will focus on the \textbf{atomization} processes.

There are several atomization processes, with the most commercially relevant being~\cite{aromaUltrasonicVsNebul}:
\begin{item-c}
\item \emph{Ultrasonic diffusers}: it contains reservoirs for water and
  essential aromatic oils. It uses mechanical ultrasonic vibrations to
  brake down water molecules into droplets, producing mist, diffusing the oils
  into the air. Its advantages are: low power consumption, easy to clean,
  silent operation, and
  they double as a humidifier (can be a disadvantage too). Furthermore, they are
  a very cost-effective solution: the units themselves tend to cost less than
  nebulizing diffusers on average, but more importantly, ultrasonic diffusers
  use much less oil than nebulizing diffusers. They also run for longer periods
  of time, in several cases up to 24 hours before needing to be refilled.
  As a disadvantage they change the fragrance composition by incorporating water
  into it (this is not critical).
\item \emph{Nebulizers}:
  Nebulizing diffusers don't use water. Instead the essential oil is diffused by
  an air compressor that blows air across the top of the reservoir tube,
  creating a vacuum which pulls fine particles of the essential oil up and
  sprays them into the air around the unit. Its advantages are: fragrance
  composition is unaltered, more compact (typically), faster and more
  concentrated fragrance diffusion. The drawbacks are: less cost-effective when
  compared to ultrasonic diffusers as the oil consumption rate is much higher
  and the units are more expensive, and they tend to be noisy due to the air
  compressor operation.
\end{item-c}

The ultrasonic diffuser was preferred due to its low power consumption, silent
operation, cost-effectiveness, and easy control and assembly. The ultrasonic
diffusion process will be detailed in the following section.

\subsection{Ultrasonic diffusion}
\label{sec:ultrasonic-diffusion}




%%% Local Variables:
%%% mode: latex
%%% TeX-master: "../../../dissertation"
%%% End:
