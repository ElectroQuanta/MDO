%
\section{Computer vision}
\label{sec:computer-vision}
Computer vision is a vast field, but can broadly be defined as the
transformation of data from a still image or video camera into
either a decision or a new representation to achieve some particular goal~\cite{kaehler2016learning}.

The input data may include some contextual information
such as `the camera is mounted in a car' or `laser range finder indicates an
object is 1 meter away.'
The decision might be `there is a person in this scene' or `there are 14
tumor cells on this slide.'
A new representation might mean turning a color image
into a grayscale image or removing camera motion from an image
sequence~\cite{kaehler2016learning}.

In this section, computer vision frameworks/libraries are listed, with special
focus on OpenCV, and computer vision algorithms for face detection and hand
gesture recognition are analyzed.
%
\subsection{OpenCV}
\label{sec:opencv}

\subsection{Face detection}
\label{sec:face-detection}
Face detection has been studied for decades in the computer
vision literature. Modern face detection algorithms can
be categorized into four categories:
cascade based methods [2, 10, 15, 16, 21],
part based methods [19, 23, 30], channel feature based methods [25, 24], and
neural network based methods [6, 14, 25, 28].

Here we highlight a few notable studies.
A detailed survey can be found in [27, 29].
The seminal work by Viola and Jones [21] introduces inte-
gral image to compute Haar-like features in constant time.
These features are then used to learn AdaBoost classifier
with cascade structure for face detection. Various later stud-
ies follow a similar pipeline. Among those variants, SURF
cascade [15] achieves competitive performance. Chen et al. [2] learn face
detection and alignment jointly in the same cascade framework and obtain promising detection performance.

One of the well-known part based methods is deformable
part models (DPM) [7]. Deformable part models define
face as a collection of parts and model the connections
of parts through Latent Support Vector Machine. The

\subsection{Hand gesture recognition}
\label{sec:hand-gest-recogn}


%%% Local Variables:
%%% mode: latex
%%% TeX-master: "../../../dissertation"
%%% End:
