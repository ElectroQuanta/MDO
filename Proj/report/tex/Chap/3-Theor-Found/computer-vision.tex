%
\section{Computer vision}
\label{sec:computer-vision}
Computer vision is a vast field, but can broadly be defined as the
transformation of data from a still image or video camera into
either a decision or a new representation to achieve some particular goal~\cite{kaehler2016learning}.

The input data may include some contextual information
such as `the camera is mounted in a car' or `laser range finder indicates an
object is 1 meter away.'
The decision might be `there is a person in this scene' or `there are 14
tumor cells on this slide.'
A new representation might mean turning a color image
into a grayscale image or removing camera motion from an image
sequence~\cite{kaehler2016learning}.

In this section, computer vision frameworks/libraries are listed, with special
focus on OpenCV, and computer vision algorithms for face detection and hand
gesture recognition are analyzed.
%

\subsection{Computer vision frameworks}
\label{sec:comp-visi-fram}
There are several noteworthy computer vision frameworks, namely~\cite{cv-frameworks-2020}:
\begin{enum-c}
\item \emph{Google Cloud's Vision API}:
it is an easy-to-use image recognition technology that
lets developers understand the content of an image by applying powerful machine
learning models. It enables key vision
detection features within an application ---  face, and landmark
detection, image labeling, \gls{ocr}, and explicit
content tagging --- and image classification into millions of predefined
categories.
\item \emph{YOLOv3}:
YOLO (You Only Look Once) is a state-of-the-art, real-time object detection
system among the most widely used deep learning-based object detection methods. It considers object detection as a regression problem,
directly predicting the class probabilities and bounding box offsets from full
images with a single feed-forward \gls{cnn}.
YOLOv3 eliminates region proposal generation and feature
resampling and encapsulates all stages in a single network to form a true
end-to-end detection system.
\item \emph{TensorFlow}:
it is a free, open-source framework for creating algorithms to develop a
user-friendly Graphical Framework called TensorFlow Graphical Framework
(TF-GraF) for object detection API, which is widely applied to solve complex
tasks efficiently in agriculture, engineering, and medicine.
The TF-GraF provides independent virtual environments for amateurs and beginners
to design, train, and deploy machine intelligence models without coding or
\gls{cli} in the client-side.
\item \emph{libfacedetection}:
it is an open-source library for face detection in images. It uses a pre-trained
\gls{cnn}, enabling face detection on inputs with a size greater than 10×10
pixels. The source code is not dependant on any other libraries. A C++ compiler
is required to compile the source under various platforms, such as Windows,
Linux, ARM, etc..
\item \emph{Raster Vision}:
it is an open-source Python framework to build computer vision models on
satellite, aerial, and other large sets of images (including oblique drone
imagery), using deep learning or machine learning models.
It has built-in support for chip classification, object detection, and semantic
segmentation with backbends using PyTorch and Tensorflow.
The framework is also extensible to new data sources, tasks (e.g., object
detection), backend (e.g., TF Object Detection API), and cloud providers.
\item \emph{SOD}:
it is an embedded, modern cross-platform computer vision and machine learning
software library.
It exposes a set of APIs for deep-learning, advanced media analysis, and
processing, including real-time, multi-class object detection, and model
training on embedded systems with limited computational resource and \gls{iot}
devices.
Designed for computational efficiency and with a strong focus on real-time
applications, SOD includes a comprehensive set of both classic and
state-of-the-art deep-neural networks with their pre-trained models.
Although it is open source, the pre-trained models are charged (one time fee
only -- up to 30 \gls{usd}).
\item \emph{Face\_recognition}:
  it is a facial recognition API for Python and the command line, built with
  deep learning using dlib60‘s state-of-the-art face recognition.
  The model has an accuracy of 99.38\% on the Labeled Faces in the Wild
  benchmark.
\item \emph{JeelizFaceFilter}:
it is a lightweight and robust face tracking library, designed for augmented reality face filters.
This JavaScript library can detect and track the face in real-time from the
webcam video feed captured, enabling the developers to solve computer-vision
problems directly from the browser.
The key features include face detection, face tracking, face rotation
detection, mouth opening detection, multiple face detection, and tracking, video
acquisition with \gls{hd} video ability, etc.
\item \emph{OpenCV}:
it is an open-source computer vision and machine learning software library,
built to provide a common infrastructure for computer vision applications and
accelerate the use of machine perception in commercial products.
OpenCV was designed for computational efficiency and with a strong focus on
real-time applications. It is written in optimized C++ and can take advantage of
multicore processors, with wrappers written in Python, Java, Matlab, etc., and
supporting Windows, Linux, Android and Mac OS.
The library has more than 2500 optimized algorithms, including a comprehensive
set of both classic and state-of-the-art computer vision and machine learning algorithms.
These algorithms can be used to detect and recognize faces, identify objects,
classify human actions in videos, track camera movements, track moving objects,
extract 3D models of objects and produce 3D point clouds from stereo cameras.
It can stitch images together to produce a high-resolution image of an entire
scene, find similar images from an image database, remove red eyes from images
taken using flash, follow eye movements, recognize scenery and establish markers to overlay it with augmented reality.
\end{enum-c}
%
%\subsubsection{OpenCV}
%\label{sec:opencv}
%
\subsection{Face detection}
\label{sec:face-detection}
Face detection has been studied for decades in the computer
vision literature. Modern face detection algorithms can
be categorized into four categories:
cascade based methods [2, 10, 15, 16, 21],
part based methods [19, 23, 30], channel feature based methods [25, 24], and
neural network based methods [6, 14, 25, 28].

Here we highlight a few notable studies.
A detailed survey can be found in [27, 29].
The seminal work by Viola and Jones [21] introduces inte-
gral image to compute Haar-like features in constant time.
These features are then used to learn AdaBoost classifier
with cascade structure for face detection. Various later stud-
ies follow a similar pipeline. Among those variants, SURF
cascade [15] achieves competitive performance. Chen et al. [2] learn face
detection and alignment jointly in the same cascade framework and obtain promising detection performance.

One of the well-known part based methods is deformable
part models (DPM) [7]. Deformable part models define
face as a collection of parts and model the connections
of parts through Latent Support Vector Machine. The

\subsection{Hand gesture recognition}
\label{sec:hand-gest-recogn}


%%% Local Variables:
%%% mode: latex
%%% TeX-master: "../../../dissertation"
%%% End:
