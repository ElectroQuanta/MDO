% For VIM to recognize the document syntax \begin{document} and \end{document}
% - However, the compilation will fail!! So don't forget to comment the
%   directives before compiling!!'
%
%\begin{document}
%
% CHAPTER - Introduction -------------------------
\chapter{Introduction}%
\label{ch:introduction}
The present work, within the scope of the curricular unit of Nonlinear Dynamic
Systems and Neural Networks, explores the problem of behavior generation for an
autonomous mobile robot.

An autonomous mobile vehicle must be capable of pursuing a target while
simultaneously avoiding collision with obstacles~\cite{bicho2000dynamic}. This
desired behavior for the robot's motion can be described and structured using
as behavioral variables (also known as state variables) the heading direction
relative to an external world axis, $\gls{phi}$, and the linear velocity of the
robot, $\gls{v}$. The temporal evolution of the behavioral variables is governed by
nonlinear dynamics systems, described by ordinary differential equations~\cite{bicho2000dynamic}:
%
\begin{equation}
  \label{eq:1}
  \frac{d \phi}{dt} = f (\phi{,} parameters),
\end{equation}
%
\begin{equation}
  \label{eq:2}
\frac{dv}{dt} = g (v{,}parameters)
\end{equation}
%
where the parameters depend, in general, of the sensorial information, hence the
respective vectorial fields vary --- number and position of fixed points and
respective stability --- as the robot moves or the environment changes (i.e.,
dynamic environments)~\cite{bicho2000dynamic}.

The desired values for the behavioral variables --- e.g., the direction the
target --- are made attractive force-lets (i.e., asymptotically stable states) of the
dynamic system and unsidered values --- e.g., the direction that obstacles are
detected --- are made repulsive force-lets (i.e., unstable states) of the
dynamic system. The overall design idea for the autonomous navigation is to
design such nonlinear dynamic systems such that it rests on, or very near of, an
attractive force-let, thus yielding it robust to variations of the vectorial
fields (robot motion and environment changes) and to noise.

Qualitative changes in the behavior of the robot --- e.g., pass between
obstacles or circumvent them --- are generated by inducing proper bifurcations
in the vectorial fields as a function of the parameters that depend on sensorial
information.

The goal of the present work is to analyze, implement, simulate and tune the
parameters of the dynamic systems yielding attractive and repulsive force-lets
with adequate values, and that simultaneously the state variables follows
closely one of the attractive force-lets as these vary. This is paramount to the
design of autonomous mobile robots as it contributes to the asymptotical
stability of the system that generates the robot's behavior, yielding it robust
to external disturbances.

%     % REPORT ORGANISATION
%\input{./tex/Chap/Intro/report-organisation}
\section{Report organization}%
\label{sec:report-organisation}
The present work is divided into small projects to understand and analyze the
contribution of each component to the behavior of the robot. For each project
the following tasks are performed:
\begin{enumerate}
\item Analytical study of the dynamic system
\item Implementation, in the simulator, of the dynamic system that generates the
  movement of the robot
\item Simulation and analysis of the behavior of the robot
\end{enumerate}

This report is organized as follows:
\begin{itemize}
%\item In Chapter~\ref{ch:state-art}, the state of the art of remote controlled
%vehicles is presented.
\item Chapter~\ref{ch:introduction} lays out the theoretical foundations for
  this project and its overall goal.
\item In Chapter~\ref{ch:target-acquisition} is studied, in isolation, the
  target acquisition behavior for the robot. Furthermore, it is analyzed the
  influence of the linearity of the vector field for the heading direction of
  the robot.
\item In Chapter~\ref{ch:obstacle-avoidance} is studied, in isolation, the
  obstacle avoidance behavior for the robot.
\item In Chapter~\ref{ch:obstacle-target-nonlinear}, the target acquisition and
  obstacles avoidance behaviors are integrated for fulfilling the overall goal
  of the robot --- move to target in a collision free path --- with the dynamic
  system for the target acquisition being nonlinear.
\item In Chapter~\ref{ch:obstacle-target-linear}, the behaviors are also
  integrated for the same overall goal, but with the dynamic system for the
  target acquisition being linear.
\item In Chapter~\ref{cha:contr-driv-speed}, the control of the driving speed is
  added in a more systematic way to improve the overall dynamics.
\item Chapter~\ref{ch:conclusion} gives a summary of the main findings of this
  work, defining the desired structure for the nonlinear dynamic systems of the
  behavioral variables --- heading direction and linear velocity of the robot
  --- and the range of acceptable values for the parameters and how to
  successfully implement and tune the system.
\end{itemize}
%
%%% Local Variables:
%%% mode: latex
%%% TeX-master: "../../dissertation"
%%% End:
