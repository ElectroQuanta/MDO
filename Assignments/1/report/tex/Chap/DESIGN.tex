\chapter{Design}
\label{ch:design}
In this section the theoretical foundations are used to design a viable
solution, accordingly to the requirements and constraints listed.
In the design phase, the product development starts, specifying the system in
terms of hardware and software and its associated interfaces, the error handling
required, and the design verification.
%
  \vspace{-5mm}
%  
\section{Hardware specification}
\label{sec:hw-specs}
- Block diagram with COTS components, if possible
- List of constraints of functions to be implemented in HW or SW
  - Inclusion of a multiplexer may reduce SW burden
  - CPU peripherals:
    - PCA for wave generation
    
\section{Hardware interfaces definition}
\label{sec:hw-interf-def}
- I/O ports
- HW registers
- Memory addresses for shared or I/O by memory mapping
- HW interrupts

\section{Software specification}
\label{sec:sw-specs}
Top-down methodology
1. Identify main subsystems
   1. Signal input detector
   2. Event handler
   3. Output generator

\section{Software interfaces definition}
\label{sec:sw-interf-def}
- Define the APIs in detail:
  - header files with:
    - functions prototypes
    - data structure declarations
    - class declarations

\section{Start-up/shutdown process specification}
\label{sec:startup-shutdown}

\section{Error handling specification}
\label{sec:error-handling-specification}
- Create error-handling routines
- Watchdog timer can be used for system recovery

\section{Design verification}
\label{sec:design-verification}
%%% Local Variables:
%%% mode: latex
%%% TeX-master: "../../dissertation"
%%% End:
